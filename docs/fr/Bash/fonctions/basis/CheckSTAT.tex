\documentclass[a4paper,10pt]{article}

\usepackage[utf8]{inputenc}     % Encodage du texte
\usepackage[french]{babel}      % Langue
\usepackage[sfdefault]{roboto}  % Police d'écriture utilisée dans le document.
\usepackage[T1]{fontenc}
\usepackage{color}
\usepackage{fancyhdr}

\fontfamily{}

\title{Fonctions du fichier de librairie CheckSTAT.lib}
\author{Dimitri OBEID}
\date{2021}
\pagestyle{fancy}

\pdfinfo{
  /Title    (Fonctions du fichier de librairie CheckSTAT.lib)
  /Author   (Dimitri OBEID)
  /Creator  (Dimitri OBEID)
  /Producer (Dimitri OBEID)
  /Subject  (Fonctions du fichier de librairie CheckSTAT.lib)
  /Keywords ()
}

\begin{document}
\maketitle
\newpage

\tableofcontents
\newpage

\section{Présentation}
\section{Fonctions}
\subsection{ConfEcho}
\begin{itemize}
    \item \textbf{Description de la fonction :} Cette fonction

    \item \textbf{Paramètres :} \$p\_file : fichier dans lequel cette fonction est appelée (de préférence avec la commande "\$(basename "\$BASH\_SOURCE[0]")")
    \$p\_lineno : le numéro de ligne où cette fonction est appelée (de préférence avec la variable d'environnement "\$LINENO")
    \$p\_badVal :
    \$p\_varVal :
    
    \item \textbf{Variables :} v\_array :
    i :

    \item \textbf{Code :}
\end{itemize}

\subsection{CheckSTAT\_DEBUG}
\begin{itemize}
    \item \textbf{Description de la variable :} Cette variable de statut sert à lancer un déboguage de certaines fonctionnalités (dans une section de code écrite de préférence au début du script principal), sans attendre l'atteinte à cette fonctionnalité si le code de cette dernière est écrit trop loin dans un script.
    
    Valeurs acceptées :
    \begin{itemize}
        \item \textbf{true} : Si une condition vérifie que la valeur de cette variable de statut est égale à cette valeur, alors les fonctionnalités à tester sont testées au début du script principal, avant d'interrompre l'exécution une fois les tests passés.
        \item \textbf{false} : Le script ignore les éventuels tests de fonctionnalités. 
    \end{itemize}

    \item \textbf{Description de la fonction :} Cette fonction vérifie que la valeur enregistrée dans la variable de statut "\$\_\_STAT\_DEBUG" soit bien égale à "true" ou "false".

    \item \textbf{Paramètres :} \$p\_file : fichier dans lequel cette fonction est appelée (de préférence avec la commande "\$(basename "\$BASH\_SOURCE[0]")")
    \$p\_lineno : le numéro de ligne où cette fonction est appelée (de préférence avec la variable d'environnement "\$LINENO")

    \item \textbf{Variables :} v\_array :

    \item \textbf{Code :}
\end{itemize}

\subsection{CheckSTAT\_ERROR}
\begin{itemize}
    \item \textbf{Description de la variable :} Cette variable de statut sert à déterminer la gravité d'une erreur selon le contexte. Ce choix est à déterminer par l'utilisateur, et cette variable est traitée par la fonction "HandleErrors()" du fichier de librairie "Checkings.lib".
    
    Valeurs acceptées :
    \begin{itemize}
        \item \textbf{\textit{vide}} : Sans valeur enregistrée, le script demande à l'utilisateur s'il souhaite que son exécution continue ou non (via la fonction HandleErrors()).
        \item \textbf{fatal} : Le script interrompt son exécution sans demander l'avis de l'utilisateur, car l'erreur rencontrée est jugée trop importante pour que le script continue son exécution sans problèmes.
    \end{itemize}

    \item \textbf{Description de la fonction :} Cette fonction vérifie que la valeur enregistrée dans la variable de statut "\$\_\_STAT\_ERROR" soit vide ou corresponde à la chaîne de caractères "fatal".

    \item \textbf{Paramètres :} \$p\_file : fichier dans lequel cette fonction est appelée (de préférence avec la commande "\$(basename "\$BASH\_SOURCE[0]")")
    \$p\_lineno : le numéro de ligne où cette fonction est appelée (de préférence avec la variable d'environnement "\$LINENO")

    \item \textbf{Variables :} v\_array :

    \item \textbf{Code :}
\end{itemize}

\subsection{CheckSTAT\_EXIT\_CODE}
\begin{itemize}
    \item \textbf{Description de la variable :}

    \item \textbf{Description de la fonction :} Cette fonction vérifie que la valeur enregistrée dans la variable de statut "\$\_\_STAT\_EXIT\_CODE" soit un nombre entier.

    \item \textbf{Paramètres :} \$p\_file : fichier dans lequel cette fonction est appelée (de préférence avec la commande "\$(basename "\$BASH\_SOURCE[0]")")
    \$p\_lineno : le numéro de ligne où cette fonction est appelée (de préférence avec la variable d'environnement "\$LINENO")

    \item \textbf{Variables :} v\_array :

    \item \textbf{Code :}
\end{itemize}

\subsection{CheckSTAT\_LOG}
\begin{itemize}
    \item \textbf{Description de la variable :}

    \item \textbf{Description de la fonction :} Cette fonction vérifie que la valeur enregistrée dans la variable de statut "\$\_\_STAT\_LOG" corresponde aux chaînes de caractères "true" ou "false".

    \item \textbf{Paramètres :} \$p\_file : fichier dans lequel cette fonction est appelée (de préférence avec la commande "\$(basename "\$BASH\_SOURCE[0]")")
    \$p\_lineno : le numéro de ligne où cette fonction est appelée (de préférence avec la variable d'environnement "\$LINENO")

    \item \textbf{Variables :} v\_array :

    \item \textbf{Code :}
\end{itemize}

\subsection{CheckSTAT\_LOG\_REDIRECT}
\begin{itemize}
    \item \textbf{Description de la variable :}

    \item \textbf{Description de la fonction :} Cette fonction vérifie que la valeur enregistrée dans la variable de statut "\$\_\_STAT\_LOG\_REDIRECT" soit vide ou corresponde aux chaînes de caractères "log" ou "tee".

    \item \textbf{Paramètres :} \$p\_file : fichier dans lequel cette fonction est appelée (de préférence avec la commande "\$(basename "\$BASH\_SOURCE[0]")")
    \$p\_lineno : le numéro de ligne où cette fonction est appelée (de préférence avec la variable d'environnement "\$LINENO")

    \item \textbf{Variables :} v\_array :

    \item \textbf{Code :}
\end{itemize}

\subsection{CheckSTAT\_TIME\_TXT}
\begin{itemize}
    \item \textbf{Description de la variable :} Cette variable sert à mettre en pause l'exécution du script pendant un temps bref (vue d'un message, etc...).

    \item \textbf{Description de la fonction :} Cette fonction vérifie que la valeur enregistrée dans la variable de statut "\$\_\_STAT\_TIME\_TXT" soit un nombre entier ou décimal.

    \item \textbf{Paramètres :} \$p\_file : fichier dans lequel cette fonction est appelée (de préférence avec la commande "\$(basename "\$BASH\_SOURCE[0]")")
    \$p\_lineno : le numéro de ligne où cette fonction est appelée (de préférence avec la variable d'environnement "\$LINENO")

    \item \textbf{Variables :} v\_array :

    \item \textbf{Code :}
\end{itemize}

\subsection{CheckSTAT\_USER\_OS}
\begin{itemize}
    \item \textbf{Description de la variable :}

    \item \textbf{Description de la fonction :} Cette fonction vérifie que la valeur enregistrée dans la variable de statut "\$\_\_STAT\_USER\_OS" soit

    \item \textbf{Paramètres :} \$p\_file : fichier dans lequel cette fonction est appelée (de préférence avec la commande "\$(basename "\$BASH\_SOURCE[0]")")
    \$p\_lineno : le numéro de ligne où cette fonction est appelée (de préférence avec la variable d'environnement "\$LINENO")

    \item \textbf{Variables :}

    \item \textbf{Code :}
\end{itemize}

\subsection{CheckProjectStatusVars}
\begin{itemize}
    \item \textbf{Description de la fonction :} Cette fonction

    \item \textbf{Paramètres :} \$p\_file : fichier dans lequel cette fonction est appelée (de préférence avec la commande "\$(basename "\$BASH\_SOURCE[0]")")
    \$p\_lineno : le numéro de ligne où cette fonction est appelée (de préférence avec la variable d'environnement "\$LINENO")

    \item \textbf{Variables :} Aucune.

    \item \textbf{Code :}
\end{itemize}

\end{document}
