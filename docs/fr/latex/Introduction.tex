\documentclass[a4paper,10pt]{article}

\usepackage[utf8]{inputenc}     % Encodage du texte
\usepackage[french]{babel}      % Langue
\usepackage[sfdefault]{roboto}  % Police d'écriture utilisée dans le document.
\usepackage[T1]{fontenc}
\usepackage{color}
\usepackage{fancyhdr}

\fontfamily{}

\title{Introduction à Bash Utils}
\author{Dimitri OBEID}
\date{2021}
\pagestyle{fancy}

\pdfinfo{%
  /Title    (Introduction à Bash Utils)
  /Author   (Dimitri OBEID)
  /Creator  (Dimitri OBEID)
  /Producer (Dimitri OBEID)
  /Subject  (Introduction à Bash Utils)
  /Keywords ()
}

\begin{document}
\maketitle
\newpage

\tableofcontents
\section{Introduction à Bash Utils}
\subsection{Présentation}
Bash Utils est une librairie, c'est à dire un ensemble de fonctions utilitaires, regroupées et mises à disposition afin de pouvoir être utilisées sans avoir à les réécrire.

Comme son nom le suggère, il s'agit d'une librairie orientée vers le langage Bash, un langage de script permettant une interaction simple avec le shell (interpréteur de commandes) du système d'exploitation, ainsi qu'avec ses différents programmes.

Le shell utilisé est le Bash, qui est de loin le shell le plus utilisé dans le monde d'UNIX. Ce choix s'explique par la très vaste documentation disponible, ainsi que pour sa

\subsection{Architecture}
Bash Utils est un ensemble de fichiers shell à sourcer dans un script. Ces derniers se situent dans le dossier "lib".

Chaque fichier contient des fonctions qui sont propres au nom du fichier. Par exemple, le fichier "Files.lib" ne contient que des fonctions de traitement de fichiers.

Plusieurs dossiers ont été créés pour différencier les différents fichiers :
\begin{itemize}
    \item \textbf{basis} : les fonctions les plus basiques (affichage de texte coloré, vérifications, gestion de couleurs, saisies de l'utilisateur).\\
    
    \item \textbf{main} : les fonctions standard (traitement de fichiers, de dossiers, de permissions, gestion du temps, etc...)
    
    Ces fonctions dépendent des fonctions contenues dans les fichiers du dossier "basis".\\

    \item os\_specific : les fonctions spécifiques aux systèmes UNIX supportés (installations, configurations, etc...)
\end{itemize}


Au vu du nombre de fichiers, il serait très fastidieux de mettre à jour 


\section{Installation de Bash Utils}
\section{Documentations des fonctions et des variables de Bash-Utils}

\end{document}
