\documentclass[a4paper,10pt]{article}

\usepackage[utf8]{inputenc}     % Encodage du texte.
\usepackage[french]{babel}      % Langue du document.
\usepackage[sfdefault]{roboto}  % Police d'écriture utilisée dans le document.
\usepackage[T1]{fontenc}					%
\usepackage[usenames,dvipsnames]{xcolor}	% Coloration de texte.
\usepackage{verbatim}						% Mise en page de paragraphes.
\usepackage{fancyhdr}						%
\usepackage[document]{ragged2e}								% Justification du texte.
\usepackage[a4paper,margin=1in,footskip=0.25in]{geometry}	% Mise en page du document.					%
\usepackage{hyperref}

\fontfamily{}

\pagecolor{black}
\title{\color{red}Tutoriels d'aide aux ajouts basiques}\color{white}
\author{Dimitri OBEID}
\date{2021}
\pagestyle{fancy}
\justifying

\pdfinfo{
  /Title    (Tutoriels d'aide aux ajouts basiques)
  /Author   (Dimitri OBEID)
  /Creator  (Dimitri OBEID)
  /Producer (Dimitri OBEID)
  /Subject  (Tutoriels d'aide aux ajouts basiques)
  /Keywords ()
}

% Définition de couleurs
\definecolor{mauve}{RGB}{128, 0, 128}


% TODO : Justifier chaque paragraphe du document.
\begin{document}
\maketitle
\newpage

\hypertarget{contents}{}
\tableofcontents
\newpage

\color{red}
\section{Liste des tutoriels}\color{white}

\color{green}
\subsection{Ajouter une nouvelle langue à la librairie}\color{white}

\color{blue}
\subsubsection{Fichiers à éditer}\color{white}


\color{blue}\par\noindent\rule{\textwidth}{0.4pt}\color{white}

\color{blue}
\subsubsection{Consignes}\color{white}


\color{green}\par\noindent\rule{\textwidth}{0.4pt}\color{white}

\color{green}
\subsection{Ajouter une nouvelle langue à la documentation}\color{white}

\color{blue}
\subsubsection{Fichiers à éditer}\color{white}


\color{blue}\par\noindent\rule{\textwidth}{0.4pt}\color{white}

\color{blue}
\subsubsection{Consignes}\color{white}

\end{document}

