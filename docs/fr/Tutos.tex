\documentclass[a4paper,10pt]{article}

\usepackage[utf8]{inputenc}     % Encodage du texte.
\usepackage[french]{babel}      % Langue du document.
\usepackage[sfdefault]{roboto}  % Police d'écriture utilisée dans le document.
\usepackage[T1]{fontenc}					%
\usepackage[usenames,dvipsnames]{xcolor}	% Coloration de texte.
\usepackage{verbatim}						% Mise en page de paragraphes.
\usepackage{fancyhdr}						%
\usepackage[document]{ragged2e}								% Justification du texte.
\usepackage[a4paper,margin=1in,footskip=0.25in]{geometry}	% Mise en page du document.					%
\usepackage{hyperref}                                       % Pour permettre la création de liens.

\fontfamily{}

\pagecolor{black}
\title{\color{red}Tutoriels d'aide aux ajouts basiques}\color{white}
\author{Dimitri OBEID}
\date{2021}
\pagestyle{fancy}
\justifying

\pdfinfo{
  /Title    (Tutoriels d'aide aux ajouts basiques)
  /Author   (Dimitri OBEID)
  /Creator  (Dimitri OBEID)
  /Producer (Dimitri OBEID)
  /Subject  (Tutoriels d'aide aux ajouts basiques)
  /Keywords ()
}

% Définition de couleurs
\definecolor{mauve}{RGB}{128, 0, 128}


% TODO : Justifier chaque paragraphe du document.
\begin{document}
\maketitle
\newpage

\hypertarget{contents}{}
\tableofcontents
\newpage

\color{red}
\section{Liste des tutoriels}\color{white}

\color{green}
\subsection{Ajouter une nouvelle langue à la librairie}\color{white}

\color{blue}
\subsubsection{Fichiers à éditer}\color{white}
\begin{itemize}
    \item \textbf{\color{orange}\$\_\_BU\_MODULE\_UTILS\_ROOT\color{lime}}
\end{itemize}


% -----------------------------------------------

\color{blue}\par\noindent\rule{\textwidth}{0.4pt}\color{white}

\color{blue}
\subsubsection{Consignes}\color{white}


% -----------------------------------------------

\color{green}\par\noindent\rule{\textwidth}{0.4pt}\color{white}

\color{green}
\subsection{Ajouter une nouvelle langue à la documentation}\color{white}

\color{blue}
\subsubsection{Fichiers à éditer}\color{white}
\begin{itemize}
    \item \textbf{\color{orange}\$\_\_BU\_MODULE\_UTILS\_ROOT\color{lime}}
\end{itemize}


% -----------------------------------------------

\color{blue}\par\noindent\rule{\textwidth}{0.4pt}\color{white}

\color{blue}
\subsubsection{Consignes}\color{white}


% -----------------------------------------------

\color{green}\par\noindent\rule{\textwidth}{0.4pt}\color{white}

\color{green}
\subsection{Ajouter une nouvelle couleur}\color{white}

\color{blue}
\subsubsection{Fichiers de configuration à éditer}\color{white}
\begin{itemize}
    \item \textbf{\color{orange}\$\_\_BU\_MODULE\_UTILS\_ROOT\color{lime}/config/module/main/colors.conf}
\end{itemize}


% -----------------------------------------------

\color{blue}\par\noindent\rule{\textwidth}{0.4pt}\color{white}

\color{blue}
\subsubsection{Fichiers de librairie à éditer}\color{white}
\begin{itemize}
    \item \textbf{\color{orange}\$\_\_BU\_MAIN\_ROOT\_DIR\_PATH\color{lime}/lib/function/main/Decho.lib}
    \item \textbf{\color{orange}\$\_\_BU\_MAIN\_ROOT\_DIR\_PATH\color{lime}/lib/function/main/Headers.lib}
\end{itemize}


% -----------------------------------------------

\color{blue}\par\noindent\rule{\textwidth}{0.4pt}\color{white}

\color{blue}
\subsubsection{Consignes}\color{white}
\textbf{Note :} Pour des raisons de lisibilité, veuillez créer les nouvelles fonctions et variables dans l'ordre\linebreak alphabétique de chacune de leurs liste.\\[1\baselineskip]

Dans le fichier \textbf{\color{lime}colors.conf}, repérez la section \textbf{COLOR ENCODING}, puis la sous-section \textbf{COLOR CODES FOR ENCODING}, puis créez une nouvelle variable globale de code couleur, de préférence nommée :
\begin{itemize}
    \item \textbf{\color{orange}\_\_BU\_MAIN\_COLOR\_CODE\_<\$nom\_de\_la\_couleur\_en\_anglais>}\\[1\baselineskip]
\end{itemize}
 suivie de son code de la table des couleurs de la commande \textbf{\color{gray}tput setaf}.\\[1\baselineskip]

Pour plus d'informations concernant les valeurs de cette table, veuillez cliquer sur le lien ci-dessous :\linebreak
\url{https://unix.stackexchange.com/questions/269077/tput-setaf-color-table-how-to-determine-color-codes/269085#269085}\\[1\baselineskip]

Une fois le code couleur ajouté, dans le même fichier, repérez la sous-section \textbf{ENCODING WITH THE "tput" COMMAND AND PRINTED AND REDIRECTED WITH THE "CheckTextColor" FUNCTIONS}, puis créez une nouvelle variable globale de nom de couleur nommée comme la variable créée précédement, sans le \textbf{\color{orange}\_CODE}, puis appelez la fonction \textbf{\color{mauve}CheckTextColor}, suivie du nom de la variable créée dans la sous-section \textbf{COLOR CODES FOR ENCODING}.\\[1\baselineskip]

Regardez la manière dont la fonction \textbf{\color{mauve}CheckTextColor} est appelée, pour vous aider.\\[2\baselineskip]


Maintenant, rendez vous dans le fichier \textbf{\color{orange}\$\_\_BU\_MAIN\_ROOT\_DIR\_PATH\color{lime}/lib/function/main/Decho.lib}, repérez la sous-catégorie suivante : 
\begin{itemize}
    \item \textbf{WRITING DIFFERENTLY COLORED/FORMATTED TEXT BETWEEN TEXT - LIST}\\[1\baselineskip]
\end{itemize}

Puis repérez les premières fonctions, toutes nommées \textbf{\color{mauve}Decho\_<\$nom\_de\_couleur\_en\_anglais>}, puis ajoutez une fonction pour chaque couleur en copiant simplement l'une des fonctions en la renommant \textbf{\color{mauve}Decho\_<\$nom\_de\_la\_nouvelle\_couleur>} et en remplaçant le second argument de la fonction \textbf{\color{mauve}Decho} par le nom de la variable globale de nom de couleur associée.\\[2\baselineskip]


Ensuite, rendez-vous dans le fichier \textbf{\color{orange}\$\_\_BU\_MAIN\_ROOT\_DIR\_PATH\color{lime}/lib/function/main/Headers.lib}, puis dans la sous-section \textbf{UNICOLOR HEADERS}. Repérez la liste de fonctions nommées :
\begin{itemize}
    \item \textbf{\color{mauve}Header<\$nom\_de\_la\_couleur\_en\_anglais>}\linebreak
\end{itemize}

Copiez simplement l'une des fonctions, puis renommez la \textbf{\color{mauve}Header<\$nom\_de\_la\_couleur\_en\_anglais>}, et modifiez les noms des premier et troisième arguments de la fonction \textbf{\color{mauve}HeaderBase}, en les\linebreak remplaçant par le nom de la variable globale de noms de couleurs souhaitée.\\[1\baselineskip]

Rendez vous ensuite dans la sous-section suivante, nommée \textbf{BICOLOR HEADERS}, puis pour chaque liste de fonction par couleur, copiez-collez une fonction, puis rennomez la en ne changeant que le nom de la deuxième couleur, puis le nom du troisième argument de la fonction \textbf{\color{mauve}HeaderBase}.\\[1\baselineskip]

Finalement, pour chaque couleur que vous ajoutez, créez une nouvelle liste de fonctions bicolores,\linebreak toujours en respectant l'ordre alphabétique du nom des couleurs en anglais (voir les commentaires avant chaque liste).\\[1\baselineskip]

Créez une fonction pour chacune des couleurs à supporter, nommée \textbf{\color{mauve}Header}, suivie du \textbf{\color{mauve}nom de la\linebreak nouvelle couleur}, suivie enfin du \textbf{\color{mauve}nom d'une autre couleur}.\\[1\baselineskip]

Modifiez la valeur du premier argument en appelant la variable globale de nom de couleur associée à la nouvelle couleur, puis la valeur du troisième argument en appelant la variable globale de nom de couleur dédiée au nom de l'autre couleur à afficher.\\[1\baselineskip]

Encore une fois, n'hésitez pas à jeter un œil à l'organisation des autres listes de fonctions d'affichage bicolore.\\[1\baselineskip]


% -----------------------------------------------

\color{green}\par\noindent\rule{\textwidth}{0.4pt}\color{white}

\color{green}
\subsection{\$\_TITRE}\color{white}

\color{blue}
\subsubsection{Fichiers à éditer}\color{white}
\begin{itemize}
    \item \textbf{\color{lime}}
\end{itemize}


% -----------------------------------------------

\color{blue}\par\noindent\rule{\textwidth}{0.4pt}\color{white}

\color{blue}
\subsubsection{Consignes}\color{white}


% -----------------------------------------------

\color{green}\par\noindent\rule{\textwidth}{0.4pt}\color{white}

\color{green}
\subsection{\$\_TITRE}\color{white}

\color{blue}
\subsubsection{Fichiers à éditer}\color{white}
\begin{itemize}
    \item \textbf{\color{lime}}
\end{itemize}

% -----------------------------------------------

\color{blue}\par\noindent\rule{\textwidth}{0.4pt}\color{white}

\color{blue}
\subsubsection{Consignes}\color{white}


% -----------------------------------------------

\color{green}\par\noindent\rule{\textwidth}{0.4pt}\color{white}

\color{green}
\subsection{\$\_TITRE}\color{white}

\color{blue}
\subsubsection{Fichiers à éditer}\color{white}
\begin{itemize}
    \item \textbf{\color{lime}\color{white}}
\end{itemize}


% -----------------------------------------------

\color{blue}\par\noindent\rule{\textwidth}{0.4pt}\color{white}

\color{blue}
\subsubsection{Consignes}\color{white}


% -----------------------------------------------

\color{green}\par\noindent\rule{\textwidth}{0.4pt}\color{white}

\color{green}
\subsection{\$\_TITRE}\color{white}

\color{blue}
\subsubsection{Fichiers à éditer}\color{white}
\begin{itemize}
    \item \textbf{\color{lime}\color{white}}
\end{itemize}


% -----------------------------------------------

\color{blue}\par\noindent\rule{\textwidth}{0.4pt}\color{white}

\color{blue}
\subsubsection{Consignes}\color{white}


% -----------------------------------------------

\color{green}\par\noindent\rule{\textwidth}{0.4pt}\color{white}

\end{document}

