\documentclass[a4paper,10pt]{article}

\usepackage[utf8]{inputenc}     % Encodage du texte.
\usepackage[french]{babel}      % Langue du document.
\usepackage[sfdefault]{roboto}  % Police d'écriture utilisée dans le document.
\usepackage[T1]{fontenc}					%
\usepackage[usenames,dvipsnames]{xcolor}	% Coloration de texte.
\usepackage{verbatim}						% Mise en page de paragraphes.
\usepackage{fancyhdr}						%
\usepackage[document]{ragged2e}								% Justification du texte.
\usepackage[a4paper,margin=1in,footskip=0.25in]{geometry}	% Mise en page du document.					%
\usepackage{hyperref}


\fontfamily{}

\pagecolor{black}
\title{Fonctions du fichier de librairie Headers.lib}
\author{Dimitri OBEID}
\date{2021}
\pagestyle{fancy}

\pdfinfo{
  /Title    (Fonctions du fichier de librairie Headers.lib)
  /Author   (Dimitri OBEID)
  /Creator  (Dimitri OBEID)
  /Producer (Dimitri OBEID)
  /Subject  (Fonctions du fichier de librairie Headers.lib)
  /Keywords ()
}

\definecolor{mauve}{RGB}{128, 0, 128}

\begin{document}
 \maketitle
 \tableofcontents
 \newpage

\color{red}
\section{Présentation}\color{white}

\color{red}
\section{Fonctions}\color{white}

\color{green}
\subsection{Création de base d'un header}\color{white}

\color{mauve}\textbf{DrawLine}\color{white}\linebreak
\textbf{Description :}\linebreak
Cette fonction dessine une ligne en remplissant chaque colonne du terminal avec un caractère choisi et passé en second argument.\linebreak

\textbf{Paramètres :}\linebreak
\color{orange}\textbf{p\_lineColor }\color{white} --> Ce paramètre attend un nombre entier, correspondant à un code couleur.\linebreak
\color{orange}\textbf{p\_lineChar }\color{white} --> Ce paramètre attend un seul caractère, qui sera affiché sur chaque colonne.\linebreak

\textbf{Fonctionnement :}\linebreak
La fonction \color{mauve}\textbf{DrawLine }\color{white} vérifie d'abord qu'une valeur soit passée en premier argument (paramètre \color{orange}\textbf{p\_lineColor}\color{white}). Si c'est le cas, la commande \color{gray}\textbf{echo }\color{white} est appelée avec ses options \color{gray}\textbf{-ne }\color{white} pour afficher.\\[1\baselineskip]

\textbf{Utilisation :}\linebreak
Utilisez cette fonction pour dessiner une ligne devant remplir la totalité des colonnes d'une ligne du terminal.\\[1\baselineskip]

\color{mauve}\textbf{HeaderBase}\color{white}\linebreak
\textbf{Description :}\linebreak
Cette fonction affiche un header complet.\\[1\baselineskip]

\textbf{Paramètres :}\linebreak
\color{orange}\textbf{p\_lineColor }\color{white} --> Premier paramètre servant à définir la couleur souhaitée du caractère lors de l'appel de la fonction \color{mauve}\textbf{DrawLine}\color{white}.\\[1\baselineskip]
\color{orange}\textbf{p\_lineChar }\color{white} --> Deuxième paramètre servant à définir le caractère souhaité lors de l'appel de la fonction \color{mauve}\textbf{DrawLine}\color{white}.\\[1\baselineskip]
\color{orange}\textbf{p\_stringColor }\color{white} --> .\\[1\baselineskip]
\color{orange}\textbf{p\_stringTxt }\color{white} --> .\\[1\baselineskip]

\textbf{Fonctionnement :}\linebreak
Tout d'abord, la fonction \color{mauve}\textbf{HeaderBase }\color{white} vérifie l'état de la variable de statut \color{orange}\textbf{\_\_BU\_MAIN\_STAT\_ECHO }\color{white}, vérifiant si un texte peut être redirigé vers un fichier de logs sans provoquer de boucles infinies en rappelant certaines fonctions\color{white}.\\[1\baselineskip]
\color{mauve}\textbf{HeaderBase }\color{white} appelle la fonction \color{mauve}\textbf{DrawLine }\color{white} en y passant en arguments ses deux premiers paramètres.\\[1\baselineskip]

\textbf{Utilisation :}\linebreak


\color{green}
\subsection{Headers unicolores}\color{white}
Chacune de ces fonctions 

\color{green}
\subsection{Headers multicolores}\color{white}

\end{document}
