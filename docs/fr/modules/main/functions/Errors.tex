\documentclass[a4paper,10pt]{article}

\usepackage[utf8]{inputenc}     % Encodage du texte.
\usepackage[french]{babel}      % Langue du document.
\usepackage[sfdefault]{roboto}  % Police d'écriture utilisée dans le document.
\usepackage[T1]{fontenc}					%
\usepackage[usenames,dvipsnames]{xcolor}	% Coloration de texte.
\usepackage{verbatim}						% Mise en page de paragraphes.
\usepackage{fancyhdr}						%
\usepackage[document]{ragged2e}								% Justification du texte.
\usepackage[a4paper,margin=1in,footskip=0.25in]{geometry}	% Mise en page du document.					%
\usepackage{hyperref}


\fontfamily{}

\pagecolor{black}
\title{\color{red}Fonctions du fichier de librairie \color{lime}Errors.lib}\color{white}
\author{Dimitri OBEID}
\date{2021}
\pagestyle{fancy}
\justifying

\pdfinfo{
  /Title    (Fonctions du fichier de librairie Errors.lib)
  /Author   (Dimitri OBEID)
  /Creator  (Dimitri OBEID)
  /Producer (Dimitri OBEID)
  /Subject  (Fonctions du fichier de librairie Errors.lib)
  /Keywords ()
}

\definecolor{mauve}{RGB}{128, 0, 128}

\begin{document}
 \maketitle
 \tableofcontents
 \newpage

\color{red}
\section{Présentation}\color{white}

\color{green}
\subsection{Présentation}\color{white}
Ce fichier source inclut des fonctions servant à la gestion d'erreurs.\\[1\baselineskip]

\color{green}
\subsection{Définitions des éléments mentionnés}\color{white}

\color{blue}
\subsubsection{Chemins de fichiers}\color{white}
\textbf{Nom du fichier :} \textbf{\color{lime}\color{white}}\\[1\baselineskip]
\textbf{Dossier parent :} \textbf{\color{lime}\color{white}*}\\[1\baselineskip]

\textbf{*} : Voir la variable \textbf{\color{orange}\$\_\_BU\_MAIN\_ROOT\_DIR\_PATH/lib/functions/main\color{white}} dans la sous-sous-catégorie\linebreak \textbf{\color{blue}Variables\color{white}}.\\[1\baselineskip]



\color{lime}\par\noindent\rule{\textwidth}{0.4pt}\color{white}\\[1\baselineskip]


\color{blue}
\subsubsection{Fonctions}\color{white}
\textbf{Fonction :} \textbf{\color{mauve}\color{white}}\\[1\baselineskip]

\textbf{Fichier de définition :} \textbf{\color{lime}\color{white}}\\[1\baselineskip]



\color{blue}\par\noindent\rule{\textwidth}{0.4pt}\color{white}

\color{blue}
\subsubsection{Variables globales externes et / ou variables d'environnement appelées}\color{white}
\textbf{Variable :} \textbf{\color{orange}\color{white}}\\[1\baselineskip]

\textbf{Description :} .\\[1\baselineskip]

\textbf{Fichier de définition :} \textbf{\color{lime}\color{white}}\\[1\baselineskip]




\par\noindent\rule{\textwidth}{0.4pt}\\[1\baselineskip]

\textbf{Variable :} \textbf{\color{orange}\color{white}}\\[1\baselineskip]

\textbf{Description :} .\\[1\baselineskip]

\textbf{Fichier de définition :} \textbf{\color{lime}\color{white}}\\[1\baselineskip]



\color{red}\par\noindent\rule{\textwidth}{0.4pt}\color{white}

\color{red}
\section{Fonctions}\color{white}

\color{green}
\subsection{}\color{white}

\end{document}
