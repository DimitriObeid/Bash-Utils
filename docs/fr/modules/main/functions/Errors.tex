\documentclass[a4paper,10pt]{article}

\usepackage[utf8]{inputenc}     % Encodage du texte.
\usepackage[french]{babel}      % Langue du document.
\usepackage[sfdefault]{roboto}  % Police d'écriture utilisée dans le document.
\usepackage[T1]{fontenc}					%
\usepackage[usenames,dvipsnames]{xcolor}	% Coloration de texte.
\usepackage{verbatim}						% Mise en page de paragraphes.
\usepackage{fancyhdr}						%
\usepackage[document]{ragged2e}								% Justification du texte.
\usepackage[a4paper,margin=1in,footskip=0.25in]{geometry}	% Mise en page du document.					%
\usepackage{hyperref}


\fontfamily{}

\pagecolor{black}
\title{\color{red}Fonctions du fichier de librairie \color{lime}Errors.lib}\color{white}
\author{Dimitri OBEID}
\date{2021}
\pagestyle{fancy}

\pdfinfo{
  /Title    (Fonctions du fichier de librairie Errors.lib)
  /Author   (Dimitri OBEID)
  /Creator  (Dimitri OBEID)
  /Producer (Dimitri OBEID)
  /Subject  (Fonctions du fichier de librairie Errors.lib)
  /Keywords ()
}

\definecolor{mauve}{RGB}{128, 0, 128}

\begin{document}
 \maketitle
 \tableofcontents
 \newpage

\color{red}
\section{Présentation}\color{white}

\color{green}
\subsection{Présentation}\color{white}

\begin{justify}
    Ce fichier source inclut des fonctions servant à la gestion d'erreurs.
\end{justify}


% -----------------------

% -----------------------------------------------

\color{green}\par\noindent\rule{\textwidth}{0.4pt}\color{white}

\color{green}
\subsection{Définitions des éléments mentionnés}\color{white}

\color{blue}
\subsubsection{Chemins de fichiers}\color{white}

\textbf{Nom du fichier : \color{lime}}\\[1\baselineskip]

\textbf{Dossier parent : \color{lime}}\\[1\baselineskip]

\begin{justify}
    \textbf{*} : Voir la variable \textbf{\color{orange}\$\_\_BU\_MAIN\_ROOT\_DIR\_PATH/lib/functions/main} dans la sous-sous-catégorie\linebreak \textbf{\color{blue}Variables}.
\end{justify}


% -----------------------

\color{lime}\par\noindent\rule{\textwidth}{0.4pt}\color{white}\\[1\baselineskip]


\color{blue}
\subsubsection{Fonctions}\color{white}

\textbf{Fonction : \color{mauve}}\\[1\baselineskip]

\textbf{Fichier de définition :} \textbf{\color{lime}}\\[1\baselineskip]


% -----------------------

\color{blue}\par\noindent\rule{\textwidth}{0.4pt}\color{white}

\color{blue}
\subsubsection{Variables globales externes et / ou variables d'environnement appelées}\color{white}

\textbf{Variable : \color{orange}}\\[1\baselineskip]

\textbf{Description :} .\\[1\baselineskip]

\textbf{Fichier de définition : \color{lime}}\\[1\baselineskip]


% -----------------------

\par\noindent\rule{\textwidth}{0.4pt}\\[1\baselineskip]

\textbf{Variable : \color{orange}}\\[1\baselineskip]

\textbf{Description :} .\\[1\baselineskip]

\textbf{Fichier de définition : \color{lime}}\\[1\baselineskip]


% -----------------------

% -----------------------------------------------

% /////////////////////////////////////////////////////////////////////////////////////////////// %

\color{red}\par\noindent\rule{\textwidth}{0.4pt}\color{white}

\color{red}
\section{Fonctions}\color{white}

\color{green}
\subsection{Sous-section CLASSIC ERRORS HANDLING}\color{white}

\color{blue}
\subsubsection{HandleSmallErrors}\color{white}

\begin{justify}
    \textbf{Description :}\\
    Cette fonction affiche un simple message d'avertissement ou d'erreur, selon le contexte choisi, contrairement à la fonction \textbf{\color{mauve}HandleErrors}, qui fait un rapport complet de l'erreur, et qui est à priviliégier pour de plus grosses erreurs.
\end{justify}

\begin{justify}
    \textbf{Paramètres :}

    \begin{itemize}
        \item \textbf{\color{orange}p\_type} Ce paramètre attend soit le caractère \textbf{W} (warning), soit le paramètre \textbf{E} (error) \\

        \item \textbf{\color{orange}p\_string} Ce paramètre attend la chaîne de caractère à afficher \\

        \item \textbf{\color{orange}p\_return} Ce paramètre attend les mêmes caractères que le paramètre \textbf{\color{orange}p\_type} \\

        \item \textbf{\color{orange}p\_cpls} Ce paramètre attend la chaîne de caractère \textbf{CPLS} (), pour changer la valeur de la variable globale de statut \textbf{\color{orange}\$\_\_BU\_MAIN\_STAT\_ECHO}.
    \end{itemize}
\end{justify}

\begin{justify}
    \textbf{Fonctionnement :}\\
    Cette fonction vérifie en premier lieu si la valeur \textbf{CPLS} est passée en quatrième argument, pour sauvegarder la valeur actuelle de la variable globale de statut \textbf{\color{orange}\$\_\_BU\_MAIN\_STAT\_ECHO}, et la mettre à \textbf{true}.
\end{justify}

\begin{justify}
    Passer une valeur à ce paramètre est nécessaire dans les situations où l'appel des fonctions \textbf{\color{mauve}Echo<...>()} peut provoquer une boucle infinie.
\end{justify}

\begin{justify}    
    Ensuite, elle vérifie quelle valeur a été passée en premier argument, puis affiche la chaîne de caractères passée en deuxième argument selon la valeur du premier (W = warning, donc appel de la fonction \textbf{\color{mauve}EchoWarning}, et E = error, donc appel de la fonction \textbf{\color{mauve}EchoError}).
\end{justify}

\begin{justify}
    Si la valeur du premier argument n'est ni renseignée, ni attendue, alors une de ces fonctions est aléatoirement appelée pour afficher cette chaîne de caractères.
\end{justify}

\begin{justify}
    Une fois la chaîne de caractère affichée, la fonction vérifie quelle valeur a été passée en troisième argument (\textbf{\color{orange}p\_return}), pour simplement continuer l'exécution de votre script ou pour directement arrêter son exécution.
\end{justify}

\begin{justify}
    En cas d'absence de valeur ou de présence d'une valeur inattendue, un message d'avertissement s'affiche sur le terminal, avant que l'exécution du script ne soit définitivement interrompue.
\end{justify}

\begin{justify}
    Enfin, si la valeur \textbf{CPLS} a été passée en quatrième argument, l'ancienne valeur est automatiquement réassignée à la variable globale de statut \textbf{\color{orange}\$\_\_BU\_MAIN\_STAT\_ECHO}.
\end{justify}

\begin{justify}
    \textbf{Utilisation :}
\end{justify}



\end{document}
