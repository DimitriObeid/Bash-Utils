\documentclass[a4paper,10pt]{article}

\usepackage[utf8]{inputenc}     % Encodage du texte.
\usepackage[french]{babel}      % Langue du document.
\usepackage[sfdefault]{roboto}  % Police d'écriture utilisée dans le document.
\usepackage[T1]{fontenc}					%
\usepackage[usenames,dvipsnames]{xcolor}	% Coloration de texte.
\usepackage{verbatim}						% Mise en page de paragraphes.
\usepackage{fancyhdr}						%
\usepackage[document]{ragged2e}								% Justification du texte.
\usepackage[a4paper,margin=1in,footskip=0.25in]{geometry}	% Mise en page du document.
\usepackage{hyperref}										% Table des matières cliquable.

\fontfamily{}

\pagecolor{black}
\title{\color{red}Fonctions du fichier de librairie \color{lime}Status.lib}\color{white}
\author{Dimitri OBEID}
\date{2021}
\pagestyle{fancy}

\pdfinfo{
  /Title    (Fonctions du fichier de librairie Status.lib)
  /Author   (Dimitri OBEID)
  /Creator  (Dimitri OBEID)
  /Producer (Dimitri OBEID)
  /Subject  (Fonctions du fichier de librairie Status.lib)
  /Keywords ()
}

% Définition de couleurs
\definecolor{mauve}{RGB}{128, 0, 128}


% TODO : Justifier chaque paragraphe du document.
\begin{document}
\maketitle
\newpage

\hypertarget{contents}{}
\tableofcontents
\newpage

\color{red}
\section{Présentation}\color{white}

\color{green}
\subsection{Présentation}\color{white}

\begin{justify}
    Ce fichier source inclut des fonctions servant à la gestion des valeurs des variables globales de statut : gestions d'erreurs et changement de valeur assité.
\end{justify}


% -----------------------

% -----------------------------------------------

\color{green}\par\noindent\rule{\textwidth}{0.4pt}\color{white}

\color{green}
\subsection{Définitions des éléments mentionnés}\color{white}

\color{blue}
\subsubsection{Chemins de fichiers}\color{white}

\textbf{Nom du fichier : \color{lime}Checkings.lib}\\[1\baselineskip]
\textbf{Dossier parent : \color{orange}\$\_\_BU\_MAIN\_ROOT\_DIR\_PATH\color{lime}/lib/functions/main}\\[1\baselineskip]


% -----------------------

\color{lime}\par\noindent\rule{\textwidth}{0.4pt}\color{white}\\[1\baselineskip]

\textbf{Nom du fichier : \color{lime}Colors.conf}\\[1\baselineskip]

\textbf{Dossier parent : \color{orange}\$\_\_BU\_MODULE\_UTILS\_ROOT\color{lime}/config/modules/main}\\[1\baselineskip]


% -----------------------

\color{lime}\par\noindent\rule{\textwidth}{0.4pt}\color{white}\\[1\baselineskip]

\textbf{Nom du fichier : \color{lime}Status.conf}\\[1\baselineskip]

\textbf{Dossier parent : \color{orange}\$\_\_BU\_MODULE\_UTILS\_ROOT\color{lime}/config/modules/main}\\[1\baselineskip]


% -----------------------

\color{lime}\par\noindent\rule{\textwidth}{0.4pt}\color{white}\\[1\baselineskip]

\textbf{Nom du fichier : \color{lime}Text.conf}\\[1\baselineskip]

\textbf{Dossier parent : \color{orange}\$\_\_BU\_MODULE\_UTILS\_ROOT\color{lime}/config/modules/main}\\[1\baselineskip]


% -----------------------

\color{blue}\par\noindent\rule{\textwidth}{0.4pt}\color{white}

\color{blue}
\subsubsection{Fonctions externes appelées}\color{white}

\textbf{Fonction : \color{mauve}DrawLine}\\[1\baselineskip]

\textbf{Description :} .\\[1\baselineskip]

\textbf{Fichier de définition : \color{lime}Checkings.lib}\\[1\baselineskip]


% -----------------------

\color{blue}\par\noindent\rule{\textwidth}{0.4pt}\color{white}

\color{blue}
\subsubsection{Variables globales externes et / ou variables d'environnement appelées}\color{white}

\textbf{Variable : \color{orange}\$\_\_BU\_MAIN\_COLOR...}\\[1\baselineskip]

\begin{justify}
    \textbf{Description :} Chacune des variables globales commençant par ce nom retourne un code couleu encodé lors de l'appel d'une de ces variables, grâce à l'appel simultané d'une fonction.
\end{justify}

\textbf{Fichier de définition : \color{lime}Colors.conf}\\[1\baselineskip]


% -----------------------

\color{orange}\par\noindent\rule{\textwidth}{0.4pt}\color{white}\\[1\baselineskip]

\textbf{Variable : \color{orange}\$\_\_BU\_MAIN\_PROJECT\_LOG\_FILE\_PATH}\\[1\baselineskip]

\begin{justify}
    \textbf{Description :} Cette variable enregistre le chemin du fichier de logs créé par le script d'initialisation si la valeur de la variable de statut associée est définie à \textbf{\color{gray}true}.
\end{justify}

\textbf{Fichier de définition : \color{lime}Project.conf} \\[1\baselineskip]


% -----------------------

\color{orange}\par\noindent\rule{\textwidth}{0.4pt}\color{white}\\[1\baselineskip]

\textbf{Variable : \color{orange}\$\_\_BU\_MAIN\_TXT\_CHAR\_HEADER\_LINE}\\[1\baselineskip]

\textbf{Description :}
\begin{justify}
    Cette variable contient le caractère à afficher en tant que ligne de header.
\end{justify}

\textbf{Fichier de définition : \color{lime}Text.conf}\\[1\baselineskip]


% -----------------------

% -----------------------------------------------

% /////////////////////////////////////////////////////////////////////////////////////////////// %

\color{red}\par\noindent\rule{\textwidth}{0.4pt}\color{white}

\color{red}
\section{Fonctions}\color{white}

\color{green}
\subsection{Sous-section ``CHECKING VALUES''}

\color{blue}
\subsubsection{BU::Main::Status::ConfEcho}\color{white}

\begin{justify}
\textbf{Description de la fonction :}\\
    Cette fonction retourne un message d'erreur en cas de mauvaise valeur définie dans une variable de statut.
\end{justify}

\begin{justify}
   	Ce message est structuré de la manière suivante :

   	\begin{itemize}
   		\item Le nom du fichier où s'est produite l'erreur, ainsi que la ligne, suivi de la description de l'erreur renvoyée par la fonction vérifiant la valeur de la variable de statut,\\

   		\item la valeur défectueuse enregistrée,\\

   		\item la liste des valeurs autorisées.
   	\end{itemize}
\end{justify}

\begin{justify}
    \textbf{Paramètres :}

    \begin{itemize}
        \item \color{orange}\textbf{\$p\_file\color{white} :} \color{white} fichier dans lequel cette fonction est appelée (de préférence avec la commande \textbf{\textbf{"\$(\color{gray}basename \color{white}"\color{orange}\$BASH\_SOURCE[0]\color{white}")")}}.\\

        \item \color{orange}\textbf{\$p\_lineno\color{white} :} \color{white} le numéro de ligne où cette fonction est appelée (de préférence avec la variable d'environnement \textbf{\color{orange}\$LINENO}).\\

        \item \color{orange}\textbf{\$p\_badVal\color{white} :} \color{white}\\

        \item \color{orange}\textbf{\$p\_varVal\color{white} :} \color{white}\\
    \end{itemize}
\end{justify}
    
\begin{justify}
    \textbf{Variables :}

    \begin{itemize}
        \item \color{orange}\textbf{\$v\_array\color{white} :} \color{white}\\
        \item \color{orange}\textbf{\$i\color{white} :} \color{white}
    \end{itemize}
\end{justify}

\begin{justify}
    \textbf{Code :}
    \begin{itemize}
        \item 
    \end{itemize}

\end{justify}


% -----------------------------------------------

\color{green}\par\noindent\rule{\textwidth}{0.4pt}\color{white}

\color{green}
\subsection{Sous-section "CHECKINGS"}\color{white}

\begin{justify}
    Cette section contient toutes les fonctions vérifiant que la variable de statut associée possède une valeur valide.
\end{justify}

\color{blue}
\subsubsection{CheckSTAT\_DEBUG}\color{white}

\begin{justify}
    \textbf{Description de la variable :}\\
    Cette variable de statut sert à lancer un déboguage de certaines fonctionnalités (dans une section de code écrite de préférence au début du script principal), sans attendre l'atteinte à cette fonctionnalité si le code de cette dernière est écrit trop loin dans un script.
\end{justify}

\begin{justify}
    \textbf{Valeurs acceptées :}

    \begin{itemize}
        \item \textbf{true} : Si une condition vérifie que la valeur de cette variable de statut est égale à cette valeur, alors les fonctionnalités à tester sont testées au début du script principal, avant d'interrompre l'exécution une fois les tests passés.\\

        \item \textbf{false} : Le script ignore les éventuels tests de fonctionnalités.
    \end{itemize}
\end{justify}

\begin{justify}
    \textbf{Description de la fonction :}\\
    Cette fonction vérifie que la valeur enregistrée dans la variable de statut \textbf{\color{orange}\$\_\_BU\_MAIN\_STAT\_DEBUG} soit bien égale à \textbf{true} ou \textbf{false}.\\[1\baselineskip]
\end{justify}

\begin{justify}
    \textbf{Paramètres :}

    \begin{itemize}
        \item \color{orange}\textbf{\$p\_file}\color{white} : fichier dans lequel cette fonction est appelée (de préférence avec la commande \textbf{\textbf{"\$(\color{gray}basename \color{white}"\color{orange}\$BASH\_SOURCE[0]\color{white}")")}}\\

        \item \color{orange}\textbf{\$p\_lineno}\color{white} : le numéro de ligne où cette fonction est appelée (de préférence avec la variable d'environnement \textbf{\color{orange}\$LINENO})\\
    \end{itemize}
\end{justify}

\begin{justify}
    \textbf{Variables :}

    \begin{itemize}
        \item \textbf{\color{orange}\$v\_array\color{white} :} \\[1\baselineskip]
    \end{itemize}
\end{justify}

\textbf{Code :}


% -----------------------

\color{blue}\par\noindent\rule{\textwidth}{0.4pt}\color{white}

\color{blue}
\subsubsection{CheckSTAT\_DECHO}\color{white}

\begin{justify}
    \textbf{Description de la variable :}
    Cette variable de statut sert à restreindre ou empêcher l'exécution de la fonction de décoration de texte avancée \textbf{\color{mauve}Decho}, dans les situations où l'appel de cette fonctions peut provoquer une boucle infinie.
\end{justify}

\begin{justify}
    \textbf{Description de la fonction :}\\
    Cette fonction vérifie que la valeur enregistrée dans la variable de statut \textbf{\color{orange}\$\_\_BU\_MAIN\_STAT\_DECHO} corresponde aux chaînes de caractères suivantes :\\

    \begin{itemize}
        \item \textbf{allow} : Cette valeur autorise tout formattage de texte, allant de la couleur au clignotement du texte, à la mise en gras ou en italique, à l'assombrissement, au soulignage, à la rayure ou au camouflage.\\

        \item \textbf{restrict} : Cette valeur restreint la décoration de texte à la couleur.\\

        \item \textbf{forbid} : Cette valeur interdit purement la décoration de texte et entoure simplement le texte à décorer de guillemets.
    \end{itemize}

\end{justify}

\begin{justify}
    \textbf{Paramètres :}

    \begin{itemize}
        \item \color{orange}\textbf{\$p\_file}\color{white} : fichier dans lequel cette fonction est appelée (de préférence avec la commande \textbf{"\$(\color{gray}basename \color{white}"\color{orange}\$BASH\_SOURCE[0]\color{white}")")}\\

        \item \color{orange}\textbf{\$p\_lineno}\color{white} : le numéro de ligne où cette fonction est appelée (de préférence avec la variable d'environnement \textbf{\color{orange}\$LINENO}))
    \end{itemize}
\end{justify}

\begin{justify}
    \textbf{Variables :}

    \begin{itemize}
        \item \textbf{\color{orange}\$v\_array\color{white} :}
    \end{itemize}
\end{justify}

\begin{justify}
    \textbf{Code :}
\end{justify}



% -----------------------

\color{blue}\par\noindent\rule{\textwidth}{0.4pt}\color{white}

\color{blue}
\subsubsection{CheckSTAT\_ECHO}\color{white}

\begin{justify}
    \textbf{Description de la variable :}\\
    Cette variable de statut sert à empêcher l'exécution de la fonction \textbf{\color{mauve}CheckProjectLogStatus}, notamment dans le cas où une fonction d'affichage de texte formaté est appelée à l'intérieur de cette fonction, où via une autre fonction appelée dans le même cas.
\end{justify}

\begin{justify}
    \textbf{Description de la fonction :}\\
    Cette fonction vérifie que la valeur enregistrée dans la variable de statut \textbf{\color{orange}\$\_\_BU\_MAIN\_STAT\_ECHO} corresponde aux chaînes de caractères \textbf{true} ou \textbf{false}.
\end{justify}

\begin{justify}
    \textbf{Paramètres :}

    \begin{itemize}
        \item \color{orange}\textbf{\$p\_file}\color{white} : fichier dans lequel cette fonction est appelée (de préférence avec la commande \textbf{"\$(\color{gray}basename \color{white}"\color{orange}\$BASH\_SOURCE[0]\color{white}")")}

        \item \color{orange}\textbf{\$p\_lineno}\color{white} : le numéro de ligne où cette fonction est appelée (de préférence avec la variable d'environnement \textbf{\color{orange}\$LINENO}))
    \end{itemize}
\end{justify}

\begin{justify}
    \textbf{Variables :}

    \begin{itemize}
        \item \textbf{\color{orange}\$v\_array\color{white} :}
    \end{itemize}
\end{justify}

\begin{justify}
    \textbf{Code :}
\end{justify}


% -----------------------

\color{blue}\par\noindent\rule{\textwidth}{0.4pt}\color{white}

\color{blue}
\subsubsection{CheckSTAT\_ERROR}\color{white}

\begin{justify}
    \textbf{Description de la variable :}\\
    Cette variable de statut sert à déterminer la gravité d'une erreur selon le contexte. Ce choix est à déterminer par l'utilisateur, et cette variable est traitée par la fonction \textbf{\color{mauve}BU::Main::Errors::HandleErrors()} du fichier de librairie \textbf{\color{lime}Checkings.lib}.
\end{justify}

\begin{justify}
    \textbf{Valeurs acceptées :}

    \begin{itemize}
        \item \textbf{\textit{vide}} : Sans valeur enregistrée, le script demande à l'utilisateur s'il souhaite que son exécution
        continue ou non (via la fonction \textbf{\color{mauve}BU::Main::Errors::HandleErrors()}).\\

        \item \textbf{fatal} : Le script interrompt son exécution sans demander l'avis de l'utilisateur, car l'erreur rencontrée est jugée trop importante pour que le script continue son exécution sans problèmes.
    \end{itemize}
\end{justify}

\begin{justify}
    \textbf{Description de la fonction :}\\
        Cette fonction vérifie que la valeur enregistrée dans la variable de statut \textbf{\color{orange}\$\_\_BU\_MAIN\_STAT\_ERROR} soit vide ou corresponde à la chaîne de caractères \textbf{fatal}.
\end{justify}

\begin{justify}
    \textbf{Paramètres :}
    \begin{itemize}
        \item \textbf{\color{orange}\$p\_file\color{white}:} fichier dans lequel cette fonction est appelée (de préférence avec la commande \textbf{\textbf{"\$(\color{gray}basename \color{white}"\color{orange}\$BASH\_SOURCE[0]\color{white}")")}}\\

        \item \color{orange}\textbf{\$p\_lineno}\color{white} : le numéro de ligne où cette fonction est appelée (de préférence avec la variable d'environnement \textbf{\color{orange}\$LINENO})
    \end{itemize}
\end{justify}

\begin{justify}
    \textbf{Variables :}

    \begin{itemize}
        \item \textbf{\color{orange}\$v\_array\color{white} :}
    \end{itemize}
\end{justify}

\begin{justify}
    \textbf{Code :}
\end{justify}

% -----------------------

\color{blue}\par\noindent\rule{\textwidth}{0.4pt}\color{white}

\color{blue}
\subsubsection{CheckSTAT\_INITALIZING}\color{white}

\begin{justify}
    \textbf{Description de la variable :}\\
    Cette variable de statut sert à indiquer si la librairie est en phase d'initialisation des modules.
\end{justify}

\begin{justify}
    \textbf{Description de la fonction :}\\
    Cette fonction vérifie que la valeur enregistrée dans la variable de statut \textbf{\color{orange}\$\_\_BU\_MAIN\_STAT\_INITIALIZING} corresponde aux chaînes de caractères \textbf{true} ou \textbf{false}.
\end{justify}

\begin{justify}
    \textbf{Paramètres :}

    \begin{itemize}
        \item \color{orange}\textbf{\$p\_file}\color{white} : fichier dans lequel cette fonction est appelée (de préférence avec la commande \textbf{"\$(\color{gray}basename \color{white}"\color{orange}\$BASH\_SOURCE[0]\color{white}")")}\\

        \item \color{orange}\textbf{\$p\_lineno}\color{white} : le numéro de ligne où cette fonction est appelée (de préférence avec la variable d'environnement \textbf{\color{orange}\$LINENO}))
    \end{itemize}
\end{justify}

\begin{justify}
    \textbf{Variables :}

    \begin{itemize}
        \item \textbf{\color{orange}\$v\_array\color{white} :}
    \end{itemize}
\end{justify}

\begin{justify}
    \textbf{Code :}
\end{justify}


% -----------------------

\color{blue}\par\noindent\rule{\textwidth}{0.4pt}\color{white}

\color{blue}
\subsubsection{CheckSTAT\_LOG}\color{white}

\begin{justify}
    \textbf{Description de la variable :}\\
    Cette variable de statut sert à autoriser ou non la création d'un fichier de logs pour le projet.
\end{justify}

\begin{justify}
    \textbf{Description de la fonction :}\\
    Cette fonction vérifie que la valeur enregistrée dans la variable de statut \textbf{\color{orange}\$\_\_BU\_MAIN\_STAT\_LOG} corresponde aux chaînes de caractères \textbf{true} ou \textbf{false}.\\[1\baselineskip]
\end{justify}

\begin{justify}
    \textbf{Paramètres :}
    \begin{itemize}
        \item \color{orange}\textbf{\$p\_file}\color{white} : fichier dans lequel cette fonction est appelée (de préférence avec la commande \textbf{"\$(\color{gray}basename \color{white}"\color{orange}\$BASH\_SOURCE[0]\color{white}")")}\\

        \item \color{orange}\textbf{\$p\_lineno}\color{white} : le numéro de ligne où cette fonction est appelée (de préférence avec la variable d'environnement \textbf{\color{orange}\$LINENO}))
    \end{itemize}
\end{justify}

\begin{justify}
    \textbf{Variables :}

    \begin{itemize}
        \item \textbf{\color{orange}\$v\_array\color{white} :}
    \end{itemize}
\end{justify}

\begin{justify}
    \textbf{Code :}
\end{justify}


% -----------------------

\color{blue}\par\noindent\rule{\textwidth}{0.4pt}\color{white}

\color{blue}
\subsubsection{CheckSTAT\_LOG\_REDIRECT}\color{white}

\begin{justify}
    \textbf{Description de la variable :}\\

\end{justify}

\begin{justify}
    \textbf{Description de la fonction :}\\
    Cette fonction vérifie que la valeur enregistrée dans la variable de statut \textbf{\color{orange}\$\_\_BU\_MAIN\_STAT\_LOG\_REDIRECT} soit vide ou corresponde aux chaînes de caractères \textbf{log} ou \textbf{tee}.
\end{justify}

\begin{justify}
    \textbf{Paramètres :}
    \begin{itemize}
        \item \color{orange}\textbf{\$p\_file}\color{white} : fichier dans lequel cette fonction est appelée (de préférence avec la commande \textbf{"\$(\color{gray}basename \color{white}"\color{orange}\$BASH\_SOURCE[0]\color{white}")")}\\

        \item \color{orange}\textbf{\$p\_lineno}\color{white} : le numéro de ligne où cette fonction est appelée (de préférence avec la variable d'environnement \textbf{\color{orange}\$LINENO}))
    \end{itemize}
\end{justify}

\begin{justify}
    \textbf{Variables :}

    \begin{itemize}
        \item \textbf{\color{orange}\$v\_array\color{white} :}
    \end{itemize}
\end{justify}

\begin{justify}
    \textbf{Code :}
\end{justify}


% -----------------------

\color{blue}\par\noindent\rule{\textwidth}{0.4pt}\color{white}

\color{blue}
\subsubsection{CheckSTAT\_OPERATE\_ROOT}\color{white}

\begin{justify}
    \item \textbf{Description de la variable :}\\

\end{justify}

\begin{justify}
    \textbf{Description de la fonction :}\\
    Cette fonction vérifie que la valeur enregistrée dans la variable de statut\\ \textbf{\color{orange}\$\_\_BU\_MAIN\_STAT\_OPERATE\_ROOT} corresponde aux chaînes de caractères \textbf{true} ou \textbf{false}.
\end{justify}

\begin{justify}
    \textbf{Paramètres :}

    \begin{itemize}
        \item \color{orange}\textbf{\$p\_file}\color{white} : fichier dans lequel cette fonction est appelée (de préférence avec la commande \textbf{"\$(\color{gray}basename \color{white}"\color{orange}\$BASH\_SOURCE[0]\color{white}")")}\\

        \item \color{orange}\textbf{\$p\_lineno}\color{white} : le numéro de ligne où cette fonction est appelée (de préférence avec la variable d'environnement \textbf{\color{orange}\$LINENO}))
    \end{itemize}
\end{justify}


\begin{justify}
    \textbf{Variables :}

    \begin{itemize}
        \item \textbf{\color{orange}\$v\_array\color{white} :}
    \end{itemize}
\end{justify}

\begin{justify}
    \textbf{Code :}
\end{justify}


% -----------------------

\color{blue}\par\noindent\rule{\textwidth}{0.4pt}\color{white}

\color{blue}
\subsubsection{CheckSTAT\_PRINT\_INIT\_LOG}\color{white}

\begin{justify}
    \textbf{Description de la variable :}\\
    Cette variable de statut détermine l'autorisation pour la librairie d'afficher les messages d'initialisation sur le terminal lors de cette procédure.
\end{justify}

\begin{justify}
    \textbf{Description de la fonction :}\\
    Cette fonction vérifie que la valeur enregistrée dans la variable de statut\\ \textbf{\color{orange}\$\_\_BU\_MAIN\_STAT\_PRINT\_INIT\_LOG} corresponde aux chaînes de caractères \textbf{true} ou \textbf{false}.
\end{justify}

\begin{justify}
    \textbf{Paramètres :}

    \begin{itemize}
        \item \color{orange}\textbf{\$p\_file}\color{white} : fichier dans lequel cette fonction est appelée (de préférence avec la commande \textbf{"\$(\color{gray}basename \color{white}"\color{orange}\$BASH\_SOURCE[0]\color{white}")")}\\

        \item \color{orange}\textbf{\$p\_lineno}\color{white} : le numéro de ligne où cette fonction est appelée (de préférence avec la variable d'environnement \textbf{\color{orange}\$LINENO}))
    \end{itemize}
\end{justify}

\begin{justify}
    \textbf{Variables :}

    \begin{itemize}
        \item \textbf{\color{orange}\$v\_array\color{white} :}
    \end{itemize}
\end{justify}

\begin{justify}
    \textbf{Code :}
\end{justify}


% -----------------------

\color{blue}\par\noindent\rule{\textwidth}{0.4pt}\color{white}

\color{blue}
\subsubsection{CheckSTAT\_TIME\_TXT}\color{white}

\begin{justify}
    \textbf{Description de la variable :}\\
    Cette variable sert à mettre en pause l'exécution du script pendant un temps bref (vue d'un message, etc...).
\end{justify}

\begin{justify}
    \textbf{Description de la fonction :}\\
    Cette fonction vérifie que la valeur enregistrée dans la variable de statut \textbf{\color{orange}\$\_\_BU\_MAIN\_STAT\_TIME\_TXT} soit un nombre entier ou décimal.
\end{justify}

\begin{justify}
    \textbf{Paramètres :}

    \begin{itemize}
        \item \color{orange}\textbf{\$p\_file}\color{white} : fichier dans lequel cette fonction est appelée (de préférence avec la commande \textbf{"\$(\color{gray}basename \color{white}"\color{orange}\$BASH\_SOURCE[0]\color{white}")")}\\

        \item \color{orange}\textbf{\$p\_lineno}\color{white} : le numéro de ligne où cette fonction est appelée (de préférence avec la variable d'environnement \textbf{\color{orange}\$LINENO}))
    \end{itemize}
\end{justify}

\begin{justify}
    \textbf{Variables :}

    \begin{itemize}
        \item \textbf{\color{orange}\$v\_array\color{white} :}
    \end{itemize}
\end{justify}

\begin{justify}
    \textbf{Code :}
\end{justify}


% -----------------------

\color{blue}\par\noindent\rule{\textwidth}{0.4pt}\color{white}

\color{blue}
\subsubsection{CheckSTAT\_TRANSLATED}\color{white}

\begin{justify}
    \textbf{Description de la variable :}\\
    Cette variable de statut .
\end{justify}

\begin{justify}
    \textbf{Description de la fonction :}\\
    Cette fonction vérifie que la valeur enregistrée dans la variable de statut \textbf{\color{orange}\$\_\_BU\_MAIN\_STAT\_PRINT\_INIT\_LOG} corresponde aux chaînes de caractères \textbf{true} ou \textbf{false}.
\end{justify}

\begin{justify}
    \textbf{Paramètres :}

    \begin{itemize}
        \item \color{orange}\textbf{\$p\_file}\color{white} : fichier dans lequel cette fonction est appelée (de préférence avec la commande \textbf{"\$(\color{gray}basename \color{white}"\color{orange}\$BASH\_SOURCE[0]\color{white}")")}\\

        \item \color{orange}\textbf{\$p\_lineno}\color{white} : le numéro de ligne où cette fonction est appelée (de préférence avec la variable d'environnement \textbf{\color{orange}\$LINENO}))
    \end{itemize}
\end{justify}

\begin{justify}
    \textbf{Variables :}

    \begin{itemize}
        \item \textbf{\color{orange}\$v\_array\color{white} :}
    \end{itemize}
\end{justify}

\begin{justify}
    \textbf{Code :}
\end{justify}


% -----------------------

\color{blue}\par\noindent\rule{\textwidth}{0.4pt}\color{white}

\color{blue}
\subsubsection{CheckSTAT\_TXT\_FMT}\color{white}

\begin{justify}
    \textbf{Description de la variable :}\\
    Cette variable de statut .
\end{justify}

\begin{justify}
    \textbf{Description de la fonction :}\\
    Cette fonction vérifie que la valeur enregistrée dans la variable de statut \textbf{\color{orange}\$\_\_BU\_MAIN\_STAT\_PRINT\_INIT\_LOG} corresponde aux chaînes de caractères \textbf{true} ou \textbf{false}.
\end{justify}

\begin{justify}
    \textbf{Paramètres :}

    \begin{itemize}
        \item \color{orange}\textbf{\$p\_file}\color{white} : fichier dans lequel cette fonction est appelée (de préférence avec la commande \textbf{"\$(\color{gray}basename \color{white}"\color{orange}\$BASH\_SOURCE[0]\color{white}")")}\\

        \item \color{orange}\textbf{\$p\_lineno}\color{white} : le numéro de ligne où cette fonction est appelée (de préférence avec la variable d'environnement \textbf{\color{orange}\$LINENO}))
    \end{itemize}
\end{justify}

\begin{justify}
    \textbf{Variables :}

    \begin{itemize}
        \item \textbf{\color{orange}\$v\_array\color{white} :}
    \end{itemize}
\end{justify}

\begin{justify}
    \textbf{Code :}
\end{justify}


% -----------------------

\color{blue}\par\noindent\rule{\textwidth}{0.4pt}\color{white}

\color{blue}
\subsubsection{CheckSTAT\_USER\_OS}\color{white}

\begin{justify}
    \textbf{Description de la variable :}\\
    
\end{justify}

\begin{justify}
    \textbf{Description de la fonction :}\\
    Cette fonction vérifie que la valeur enregistrée dans la variable de statut "\$\_\_BU\_MAIN\_STAT\_USER\_OS" soit
\end{justify}

\begin{justify}
    \textbf{Paramètres :}

    \begin{itemize}
        \item \color{orange}\textbf{\$p\_file}\color{white} : fichier dans lequel cette fonction est appelée (de préférence avec la commande \textbf{"\$(\color{gray}basename \color{white}"\color{orange}\$BASH\_SOURCE[0]\color{white}")")}\\

        \item \color{orange}\textbf{\$p\_lineno}\color{white} : le numéro de ligne où cette fonction est appelée (de préférence avec la variable d'environnement \textbf{\color{orange}\$LINENO}))
    \end{itemize}
\end{justify}

\begin{justify}
    \textbf{Variables :}

    \begin{itemize}
        \item \textbf{\color{orange}\$v\_array\color{white} :}
    \end{itemize}
\end{justify}

\begin{justify}
    \textbf{Code :}
    
    \begin{itemize}
        \item \textbf{}
    \end{itemize}
\end{justify}


% -----------------------

\color{blue}\par\noindent\rule{\textwidth}{0.4pt}\color{white}

\color{blue}
\subsubsection{CheckProjectStatusVars}\color{white}

\begin{justify}
    \textbf{Description de la fonction :}\\
    Cette fonction
\end{justify}

\begin{justify}
    \textbf{Paramètres :}
    \begin{itemize}
        \item \color{orange}\textbf{\$p\_file}\color{white} : fichier dans lequel cette fonction est appelée (de préférence avec la commande \textbf{"\$(\color{gray}basename \color{white}"\color{orange}\$BASH\_SOURCE[0]\color{white}")")}\\

        \item \color{orange}\textbf{\$p\_lineno}\color{white} : le numéro de ligne où cette fonction est appelée (de préférence avec la variable d'environnement \textbf{\color{orange}\$LINENO}))
    \end{itemize}
\end{justify}

\begin{justify}
    \textbf{Variables :}

    \begin{itemize}
        \item \textbf{Aucune}
    \end{itemize}
\end{justify}

\begin{justify}
    \textbf{Code :}
\end{justify}


% -----------------------

% -----------------------------------------------

\color{green}\par\noindent\rule{\textwidth}{0.4pt}\color{white}

\color{green}
\subsection{Sous-section "CHANGING VALUES MORE EASILY"}\color{white}

\begin{justify}
    \textbf{Note 1 :} Cette sous-section contient des fonctions appelant l'une des fonctions de la sous-section précédente, pour réduire la charge de travail du développeur lorsqu'il souhaite modifier la valeur d'une variable de statut tout en assurant le fait qu'elle contienne une valeur correcte.
\end{justify}

\begin{justify}
    \textbf{Note 2 :} Chaque fonction définie dans cette sous-section attend les trois mêmes paramètres positionnels, dont les informations sont données en commentaire sous le nom de la sous-section.

    \begin{itemize}
        \item \textbf{\$1 :} Ce paramètre contient la nouvelle valeur à assigner à la variable globale de statut traitée par la fonction.\\

        \item \textbf{\$2 :} Ce paramètre contient le chemin du fichier où le changement de valeur a été effectué. Il s'agit du premier argument de chaque fonction \textbf{\color{mauve}CheckSTAT\_} définie dans la sous-section précédente.\\

        \item \textbf{\$3 :} Ce paramètre contient le numéro de la ligne où le changement de valeur a été effectué. Il s'agit du second argument de chaque fonction \textbf{\color{mauve}CheckSTAT\_} définie dans la sous-section précédente.
    \end{itemize}
\end{justify}


% -----------------------

% -----------------------------------------------

\color{green}\par\noindent\rule{\textwidth}{0.4pt}\color{white}

\color{green}
\subsection{Sous-section "EASIER CHECKING BOOLEAN VALUES"}\color{white}

\begin{justify}
    Ces fonctions servent à vérifier si la valeur d'une des variables globales de statut contienne la valeur \textbf{vrai} ou la valeur \textbf{faux} (variables booléenes uniquement).
\end{justify}

\begin{justify}
    Chacune d'entre elles vérifie si la variable traitée contient la valeur \textbf{vrai}, et retourne la valeur 0 (fonction executée comme souhaitée), ou 1 le cas échéant.
\end{justify}



\end{document}
