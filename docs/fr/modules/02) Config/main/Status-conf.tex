\documentclass[a4paper,10pt]{article}

\usepackage[utf8]{inputenc}     % Encodage du texte (caractères accentués).
\usepackage[french]{babel}      % Langue du document.
\usepackage[sfdefault]{roboto}  % Police d'écriture utilisée dans le document.
\usepackage[T1]{fontenc}		% Règle de césure pour les caractères accentués (pour le compilateur LaTeX).

\usepackage{fancyhdr}           % Ajout d'en-têtes et de pieds de page à chaque page du document.
\usepackage{hyperref}           % Création de liens cliquables pointant vers d'autres parties du document.
\usepackage{parskip}            % Sauts de ligne déterminables entre chaque paragraphe.

% Ne pas oublier de modifier l'option "svgnames" (couleurs définies sur le modèle RGB, meilleur pour l'affichage numérique)
% en option "dvipsnames" (basé sur le modèle de couleur CJMB (CMYK), meilleur pour l'impression) via le script de conversion
% en format imprimable.
\usepackage[usenames,svgnames]{xcolor}      % Coloration du texte.
\usepackage{verbatim}						% Mise en page des paragraphes.

\usepackage[document]{ragged2e}								% Justification du texte.
\usepackage[a4paper,margin=1in,footskip=0.25in]{geometry}	% Mise en page du document.


% ------------------------------------------------------------------------------------------------------------------------
% Liste des couleurs définies (pour la mise en page et pour le changement de thème pour l'impression de la documentation).

% Définition de la couleur       % Normal | Imprimable   - Description

\definecolor{back}{HTML}{000000} % Noir | Blanc  - Couleur de fond du document.
\definecolor{case}{HTML}{fcff00} % Jaune         - Couleur des conditions "case".
\definecolor{cmds}{HTML}{909090} % Gris          - Couleur des noms de commandes du système et de leurs arguments.
\definecolor{cond}{HTML}{be480a} % Brique        - Couleur des conditions "if".
\definecolor{func}{HTML}{800080} % Mauve         - Couleur des fonctions définies dans chaque module du framework Bash Utils.

\definecolor{loop}{HTML}{00ffff} % Cyan          - Couleur des boucles.
\definecolor{main}{HTML}{8F00FF} % Violet        - Couleur des fonctions du script principal.
\definecolor{path}{HTML}{bfff00} % Citron vert   - Couleur des chemins de dossiers et de fichiers.
\definecolor{sec1}{HTML}{ff0000} % Rouge         - Couleur des titres principaux et de premier niveau.
\definecolor{sec2}{HTML}{00ff00} % Vert          - Couleur des titres de deuxième niveau.

\definecolor{sec3}{HTML}{0060ff} % Bleu          - Couleur des titres de troisième niveau.
\definecolor{text}{HTML}{ffffff} % Blanc | Noir  - Couleur du texte normal.
\definecolor{vars}{HTML}{FF7F00} % Orange        - Couleur des noms des paramètres et des variables.


% ------------------------------------------------------------------------
% Définition de la police d'écriture et de la couleur de fond du document.

\fontfamily{Roboto}

\pagecolor{back}


% ----------------------------------------
% Définition des informations du document.

\title{\color{sec1}Variables du fichier de configuration \\du module principal \color{path}Status.conf}\color{text}
\author{Dimitri OBEID}
\date{2023}
\pagestyle{fancy}

\pdfinfo{
  /Title    (Fonctions et variables du fichier de configuration du module principal "Status.conf")
  /Author   (Dimitri OBEID)
  /Creator  (Dimitri OBEID)
  /Producer (Dimitri OBEID)
  /Subject  (Fonctions et variables du fichier de configuration du module principal "Status.conf")
  /Keywords ()
}


% ------------------------------------------------------------
% Mise en page des paragraphes et des en-têtes de chaque page.

\setlength{\parskip}{1em}

\setlength{\headheight}{13pt}


% ------------------
% Début du document.

\begin{document}
    \maketitle
    \newpage

    \hypertarget{contents}{}
    \tableofcontents
    \newpage

    \color{sec1}
    \section{Présentation}\color{text}

    \begin{justify}
        Ce fichier de configurations liste chaque variable globale de statut utilisée par la librairie, avec sa ou ses valeurs par défaut.
    \end{justify}

    \begin{justify}
        Elles sont chargées en mémoire lors de l'inclusion du fichier \textbf{\color{path}Initialization.sh} appartenant au module principal.
    \end{justify}

    % ----------------------

    % -----------------------------------------------

    % /////////////////////////////////////////////////////////////////////////////////////////////// %

    \color{sec1}\par\noindent\rule{\textwidth}{0.4pt}\color{text}

    \color{sec1}
    \section{Liste des variables}\color{text}

    \begin{justify}
        \begin{tabular}{l|l|l}
            \textbf{Nom de la variable} & \textbf{Type} & \textbf{Valeurs acceptées}
        \end{tabular}
    \end{justify}

    \begin{justify}
        \textit{Description de la variable :}
    \end{justify}

    % ------------

    \par\noindent\rule{\textwidth}{0.4pt}

    \begin{justify}
        \begin{tabular}{l|l|l}
            \textbf{\color{vars}\$\_\_BU\_MAIN\_STAT\_DEBUG}  & Booléen & \textbf{false} || \textbf{true} \\[1\baselineskip]
        \end{tabular}
    \end{justify}

    \begin{justify}
        Cette variable de statut sert à lancer un déboguage de certaines fonctionnalités (dans une section de code écrite de préférence au début du script principal), sans attendre l'atteinte à cette fonctionnalité si le code de cette dernière est écrit trop loin dans un script.
    \end{justify}

    % ------------

    \par\noindent\rule{\textwidth}{0.4pt}

    \begin{justify}
        \begin{tabular}{l|l|l}
            \textbf{\color{vars}\$\_\_BU\_MAIN\_STAT\_DECHO}  & Chaîne de caractères    & \textbf{authorize} || \textbf{forbid} || \textbf{restrict}\\[1\baselineskip]
        \end{tabular}
    \end{justify}

    \begin{justify}
        Cette variable de statut sert à autoriser, empêcher \textbf{OU} restreindre l'exécution de la fonction de décoration de texte avancée \textbf{\color{func}Decho}, en fonction des situations où l'appel de cette fonctions peut provoquer ou non une boucle infinie.
    \end{justify}

    % ------------

    \par\noindent\rule{\textwidth}{0.4pt}

    \begin{justify}
        \begin{tabular}{l|l|l}
            \textbf{\color{vars}\$\_\_BU\_MAIN\_STAT\_ECHO}   & Booléen      & \textbf{false} || \textbf{true}\\[1\baselineskip]
        \end{tabular}
    \end{justify}

    \begin{justify}
        Cette variable de statut sert à totalement autoriser \textbf{OU} empêcher l'exécution de la fonction \textbf{\color{func}BU.Main.Status.CheckProjectLogStatus}, notamment dans le cas où une fonction d'affichage de texte formaté est appelée à l'intérieur de cette fonction, où via une autre fonction appelée dans le même cas.
    \end{justify}

    % ------------

    \par\noindent\rule{\textwidth}{0.4pt}

    \begin{justify}
        \begin{tabular}{l|l|l}
                    \textbf{\color{vars}\$\_\_BU\_MAIN\_STAT\_ERROR}  & Chaîne de caractères    & \textit{aucune valeur} || \textbf{fatal}\\[1\baselineskip]
        \end{tabular}
    \end{justify}

    \begin{justify}
        Cette variable de statut sert à déterminer la gravité d'une erreur selon le contexte. Ce choix est à déterminer par l'utilisateur, et la valeur de cette variable est traitée par la fonction \textbf{\color{func}BU.Main.Errors.HandleErrors()} du fichier de librairie \textbf{\color{path}Checkings.lib}.
    \end{justify}

    % ------------

    \par\noindent\rule{\textwidth}{0.4pt}

    \begin{justify}
        \begin{tabular}{l|l|l}
            \textbf{\color{vars}\$\_\_BU\_MAIN\_STAT\_INITIALIZING}       & Booléen  & \textbf{false} || \textbf{true}\\[1\baselineskip]
        \end{tabular}
    \end{justify}

    \begin{justify}
        Cette variable de statut sert à déterminer si la librairie est en phase d'initialisation des modules.
    \end{justify}

    % ------------

    \par\noindent\rule{\textwidth}{0.4pt}

    \begin{justify}
        \begin{tabular}{l|l|l}
            \textbf{\color{vars}\$\_\_BU\_MAIN\_STAT\_LOG}    & Booléen      & \textbf{false} || \textbf{true}\\[1\baselineskip]
        \end{tabular}
    \end{justify}

    \begin{justify}
        Cette variable de statut sert à autoriser ou non la création d'un fichier de logs pour le projet en cours.
    \end{justify}

    % ------------

    \par\noindent\rule{\textwidth}{0.4pt}

    \begin{justify}
        \begin{tabular}{l|l|l}
            \textbf{\color{vars}\$\_\_BU\_MAIN\_STAT\_LOG\_REDIRECT}  & Chaîne de caractères & \textit{aucune valeur} || \textbf{log} || \textbf{tee}\\[1\baselineskip]
        \end{tabular}
    \end{justify}

    \begin{justify}
        Cette variable vérifie si un message peut être affiché à l'écran, redirigé vers un fichier de logs, ou les deux à la fois.
    \end{justify}

    % ------------

    \par\noindent\rule{\textwidth}{0.4pt}

    \begin{justify}
        \begin{tabular}{l|l|l}
            \textbf{\color{vars}\$\_\_BU\_MAIN\_STAT\_OPERATE\_ROOT}  & Chaîne de caractères & \textbf{authorized} || \textbf{forbidden} || \textbf{restricted}\\[1\baselineskip]
        \end{tabular}
    \end{justify}

    \begin{justify}
        Cette variable vérifie si une opération sur un fichier ou un dossier localisé à la racine du système de fichiers peut être effectuée.
    \end{justify}

    % ------------

    \par\noindent\rule{\textwidth}{0.4pt}

    \begin{justify}
        \begin{tabular}{l|l}
            \textbf{\color{vars}\$\_\_BU\_MAIN\_STAT\_TIME\_HEADER}   & Nombre décimal\\[1\baselineskip]
        \end{tabular}
    \end{justify}

    \begin{justify}
        Cette variable sert à mettre en pause l'exécution du script pendant un temps bref lors de l'affichage d'un header.
    \end{justify}

    % ------------

    \par\noindent\rule{\textwidth}{0.4pt}

    \begin{justify}
        \begin{tabular}{l|l}
            \textbf{\color{vars}\$\_\_BU\_MAIN\_STAT\_TIME\_NEWLINE}  & Nombre décimal\\[1\baselineskip]
        \end{tabular}
    \end{justify}

    \begin{justify}
        Cette variable sert à mettre en pause l'exécution du script pendant un temps bref lors de l'affichage d'un saut de ligne.
    \end{justify}

    % ------------

    \par\noindent\rule{\textwidth}{0.4pt}

    \begin{justify}
        \begin{tabular}{l|l}
            \textbf{\color{vars}\$\_\_BU\_MAIN\_STAT\_TIME\_TXT}  & Nombre décimal\\[1\baselineskip]
        \end{tabular}
    \end{justify}

    \begin{justify}
        Cette variable sert à mettre en pause l'exécution du script pendant un temps bref lors de l'affichage d'un message à l'écran.
    \end{justify}

    % ------------

    \par\noindent\rule{\textwidth}{0.4pt}

    \begin{justify}
        \begin{tabular}{l|l|l}
            \textbf{\color{vars}\$\_\_BU\_MAIN\_STAT\_TXT\_FMT}   & Booléen & \textbf{false} || \textbf{true}\\[1\baselineskip]
        \end{tabular}
    \end{justify}

    \begin{justify}
        Cette variable de statut vérifie que le script ait la permission de formater le texte (couleurs, gras, italique, soulignage, clignotement, etc...).
    \end{justify}

    % ------------

    \par\noindent\rule{\textwidth}{0.4pt}

    \begin{justify}
        \begin{tabular}{l|l|l}
            \textbf{\color{vars}\$\_\_BU\_MAIN\_STAT\_USER\_OS}   & &\\[1\baselineskip]
        \end{tabular}
    \end{justify}

    \begin{justify}
        \textbf{ATTENTION : Cette variable est dépreciée et sera déplacée dans le module \textbf{\color{path}os\_specific} dans une prochaine mise à jour}
    \end{justify}
\end{document}
