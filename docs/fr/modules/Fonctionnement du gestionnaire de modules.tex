\documentclass[a4paper,10pt]{article}

\usepackage[utf8]{inputenc}     % Encodage du texte (caractères accentués).
\usepackage[french]{babel}      % Langue du document.
\usepackage[sfdefault]{roboto}  % Police d'écriture utilisée dans le document.
\usepackage[T1]{fontenc}		% Règle de césure pour les caractères accentués (pour le compilateur LaTeX).

\usepackage{fancyhdr}           % Ajout d'en-têtes et de pieds de page à chaque page du document.
\usepackage{hyperref}           % Création de liens cliquables pointant vers d'autres parties du document.
\usepackage{parskip}            % Sauts de ligne déterminables entre chaque paragraphe.

% Ne pas oublier de modifier l'option "svgnames" (couleurs définies sur le modèle RGB, meilleur pour l'affichage numérique)
% en option "dvipsnames" (basé sur le modèle de couleur CJMB (CMYK), meilleur pour l'impression) via le script de conversion
% en format imprimable.
\usepackage[usenames,svgnames]{xcolor}      % Coloration du texte.
\usepackage{verbatim}						% Mise en page des paragraphes.

\usepackage[document]{ragged2e}								% Justification du texte.
\usepackage[a4paper,margin=1in,footskip=0.25in]{geometry}	% Mise en page du document.


% ------------------------------------------------------------------------------------------------------------------------
% Liste des couleurs définies (pour la mise en page et pour le changement de thème pour l'impression de la documentation).

% Définition de la couleur       % Normal | Imprimable   - Description

\definecolor{back}{HTML}{000000} % Noir | Blanc  - Couleur de fond du document.
\definecolor{case}{HTML}{fcff00} % Jaune         - Couleur des conditions "case".
\definecolor{cmds}{HTML}{909090} % Gris          - Couleur des noms de commandes du système et de leurs arguments.
\definecolor{cond}{HTML}{be480a} % Brique        - Couleur des conditions "if".
\definecolor{func}{HTML}{800080} % Mauve         - Couleur des fonctions définies dans chaque module du framework Bash Utils.

\definecolor{loop}{HTML}{00ffff} % Cyan          - Couleur des boucles.
\definecolor{main}{HTML}{8F00FF} % Violet        - Couleur des fonctions du script principal.
\definecolor{path}{HTML}{bfff00} % Citron vert   - Couleur des chemins de dossiers et de fichiers.
\definecolor{sec1}{HTML}{ff0000} % Rouge         - Couleur des titres principaux et de premier niveau.
\definecolor{sec2}{HTML}{00ff00} % Vert          - Couleur des titres de deuxième niveau.

\definecolor{sec3}{HTML}{0060ff} % Bleu          - Couleur des titres de troisième niveau.
\definecolor{text}{HTML}{ffffff} % Blanc | Noir  - Couleur du texte normal.
\definecolor{vars}{HTML}{FF7F00} % Orange        - Couleur des noms des paramètres et des variables.


% ------------------------------------------------------------------------
% Définition de la police d'écriture et de la couleur de fond du document.

\fontfamily{Roboto}

\pagecolor{back}


% ----------------------------------------
% Définition des informations du document.

\title{\color{sec1}Fonctionnement du gestionnaire de modules}\color{text}
\author{Dimitri OBEID}
\date{2021}
\pagestyle{fancy}

\pdfinfo{
  /Title    (Fonctionnement du gestionnaire de modules)
  /Author   (Dimitri OBEID)
  /Creator  (Dimitri OBEID)
  /Producer (Dimitri OBEID)
  /Subject  (Fonctionnement du gestionnaire de modules)
  /Keywords ()
}


% ------------------------------------------------------------
% Mise en page des paragraphes et des en-têtes de chaque page.

\setlength{\parskip}{1em}

\setlength{\headheight}{13pt}


% ------------------
% Début du document.

\begin{document}
    \maketitle
    \tableofcontents
    \newpage

    \color{sec1}
    \section{Introduction}\color{text}

    \begin{justify}
        \textbf{Note 1 :} La procédure d'initialisation des modules et de la librairie est décrite en détails dans la sous-section \textbf{\color{sec1}Fonctionnement \color{text} / \color{sec2}Étapes d'initialisation} du fichier \textbf{\color{path}Bash-utils/docs/fr/Introduction.pdf}.
    \end{justify}

    \begin{justify}
        \textbf{Note 2 :} La méthode de création de modules est détaillée dans la section \textbf{\color{sec1}Liste des tutoriels \color{text} / \color{sec2}Ajouter un nouveau module} du fichier \textbf{\color{path}Bash-utils/docs/fr/Tutos.pdf}.
    \end{justify}

    \color{sec2}
    \subsection{Présentation générale}\color{text}

    \begin{justify}
        Ici, un module est un regroupement de fichiers source et de fichiers de configurations développé pour la librairie Bash Utils, dont le but est de faciliter le développement d'un script Bash selon une certaine catégorie.
    \end{justify}

    \begin{justify}
        Le gestionnaire de modules est un outil permettant d'initialiser chaque module et de sourcer (inclure) chaque fichier associé.
    \end{justify}

    \begin{justify}
        Pour un fonctionnement optimal, il est fortement recommandé de l'installer dans le dossier personnel de tous les utilisateurs d'un ordinateur susceptibles d'utiliser cette librairie.
    \end{justify}

    \begin{justify}
        En outre, il est obligatoire de l'installer dans le dossier personnel du super-utilisateur (le dossier \textbf{\color{path}/root}), pour pouvoir exécuter un script utilisant la librairie Bash Utils avec les privilèges du super-utilisateur, ainsi que pour utiliser certaines fonctionnalités ne pouvant être exécutées qu'avec des privilèges du super-utilisateur.
    \end{justify}

    % ------------

    % ----------------------

    % -----------------------------------------------

    \color{sec2}\par\noindent\rule{\textwidth}{0.4pt}\color{text}

    \color{sec2}
    \subsection{Dossiers}\color{text}

    \begin{justify}
        Ce gestionnaire est composé d'un dossier (caché pour ne pas trop encombrer le dossier personnel, où il doit être installé) : \textbf{\color{path}.Bash-utils}
    \end{justify}

    \begin{justify}
        Ce dossier est composé de deux sous-dossiers, comprenant également des sous dossiers : \textbf{\color{path}config} et \textbf{\color{path}modules}.
    \end{justify}

    \begin{justify}
        Le dossier \textbf{\color{path}config} est constitué du dossier \textbf{\color{path}modules}, qui contient en sous-dossiers des dossiers de modules contenant les fichiers de configuration propres à chaque module.
    \end{justify}

    \begin{justify}
        Le dossier \textbf{\color{path}modules} (sous-dossier de \textbf{\color{path}.Bash-utils}) contient en sous-dossier des dossiers de modules contenant les fichiers d'initialisation propres à chaque module.
    \end{justify}

    % ------------

    % ----------------------

    % -----------------------------------------------

    \color{sec2}\par\noindent\rule{\textwidth}{0.4pt}\color{text}

    \color{sec2}
    \subsection{Architecture}\color{text}

    \color{sec3}
    \subsubsection{Dossiers}\color{text}

    \begin{justify}

    \end{justify}
    % ------------

    % ----------------------

    \color{sec3}\par\noindent\rule{\textwidth}{0.4pt}\color{text}

    \color{sec3}
    \subsubsection{Fichiers}\color{text}

    \begin{justify}

    \end{justify}

    % ------------

    % ----------------------

    % -----------------------------------------------

    \color{sec2}\par\noindent\rule{\textwidth}{0.4pt}\color{text}

    \color{sec2}
    \subsection{Fichiers}\color{text}

    \begin{justify}
        Ce gestionnaire est composé d'un fichier à placer dans le dossier personnel de tous les utilisateurs de la librairie :  \textbf{\color{path}Bash-utils-init.sh}.
    \end{justify}

    \begin{justify}
        Parmi les fichiers propres à chaque module, deux d'entre eux doivent obligatoirement être présents :

        \begin{itemize}
            \item \textbf{\color{path}module.conf\color{text} :} Ce fichier, à placer dans le dossier \textbf{\color{path}.Bash-utils/config/modules/\$nom\_du\_module}, contient chaque variable utile au fichier \textbf{\color{path}Initializer.sh}, présenté ci-dessous.\\\mbox{}

            \item \textbf{\color{path}Initializer.sh\color{text} :} Ce fichier initialise le module et sert de fichier d'inclusion de chaque fichier source dans le dossier de librairie \textbf{\color{path}Bash-utils/lib/\$nom\_du\_module}, ainsi qu'aux autres fichiers de configuration à placer, de préférence, dans le dossier \textbf{\color{path}.Bash-utils/config/modules/}.
        \end{itemize}
    \end{justify}

    \begin{justify}
        Outre les fichiers de configuration et les fichiers source de chaque module, plusieurs fichiers sont présents directement dans le dossier \textbf{\color{path}.Bash-utils} :

        \begin{itemize}
            \item \textbf{\color{path}Bash-utils-root-val.path\color{text} :} Ce fichier contient le chemin du dossier racine de la librairie Bash Utils. Il est lu par le fichier \textbf{\color{path}module.conf}, du module \color{path}\textbf{main}.
        \end{itemize}
    \end{justify}

    % ------------

    % ----------------------

    % -----------------------------------------------

    % /////////////////////////////////////////////////////////////////////////////////////////////// %

    \color{sec1}\par\noindent\rule{\textwidth}{0.4pt}\color{text}

    \color{sec1}
    \section{Liste des modules officiels}\color{text}

    \color{sec2}
    \subsection{Main}\color{text}

    \color{sec3}
    \subsubsection{Présentation}\color{text}

    \begin{justify}
        Ce module est le module de base de la librairie. Il contient toutes les configurations et les fonctions de base de cette dernière.
    \end{justify}
    % ------------

    % ----------------------

    \color{sec3}\par\noindent\rule{\textwidth}{0.4pt}\color{text}

    \color{sec3}
    \subsubsection{Liste des fonctions}\color{text}

    \begin{justify}
        Vous pouvez retrouver ces fonctions dans les différents documents présents dans le dossier \textbf{\color{path}Bash-utils/docs/fr/modules/main/functions}
    \end{justify}% ------------

    % ----------------------

    \color{sec3}\par\noindent\rule{\textwidth}{0.4pt}\color{text}

    \color{sec3}
    \subsubsection{Liste des configurations}\color{text}

    \begin{justify}
        Vous pouvez retrouver ces fonctions dans les différents documents présents dans le dossier \textbf{\color{path}Bash-utils/docs/fr/modules/main/config}
    \end{justify}
\end{document}
