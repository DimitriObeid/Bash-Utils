\documentclass[a4paper,10pt]{article}

\usepackage[utf8]{inputenc}     % Encodage du texte (caractères accentués).
\usepackage[french]{babel}      % Langue du document.
\usepackage[sfdefault]{roboto}  % Police d'écriture utilisée dans le document.
\usepackage[T1]{fontenc}		% Règle de césure pour les caractères accentués (pour le compilateur).

\usepackage{fancyhdr}           % Ajout de headers et footers à chaque page du document.
\usepackage{hyperref}           % Création de liens cliquables pointant vers d'autres parties du document.
\usepackage{parskip}

% Ne pas oublier de modifier l'option "svgnames" (couleurs définies sur le modèle RGB, meilleur pour l'affichage numérique)
% en option "dvipsnames" (basé sur le modèle CMYK, meilleur pour l'impression) via le script de conversion en format imprimable.
\usepackage[usenames,svgnames]{xcolor}      % Coloration du texte.
\usepackage{verbatim}						% Mise en page des paragraphes.

\usepackage[document]{ragged2e}								% Justification du texte.
\usepackage[a4paper,margin=1in,footskip=0.25in]{geometry}	% Mise en page du document.


% Liste des couleurs définies (pour la mise en page et pour le changement de thème pour l'impression de la documentation).
\definecolor{back}{HTML}{000000}        % Noir          - Fond
\definecolor{case}{HTML}{fcff00}        % Jaune clair   - Conditions case
\definecolor{cmds}{HTML}{909090}        % Gris          - Commandes + arguments
\definecolor{cond}{HTML}{be480a}        % Brique        - Conditions
\definecolor{func}{HTML}{8F00FF}        % Violet        - Fonctions
\definecolor{loop}{HTML}{00ffff}        % Cyan          - Boucles
\definecolor{path}{HTML}{bfff00}        % Citron vert   - Chemins de dossiers et de fichiers
\definecolor{sec1}{HTML}{ff0000}        % Rouge         - Titres de premier niveau
\definecolor{sec2}{HTML}{00ff00}        % Vert          - Titres de deuxième niveau
\definecolor{sec3}{HTML}{0060ff}        % Bleu          - Titres de troisième niveau
\definecolor{text}{HTML}{ffffff}        % Blanc         - Texte normal
\definecolor{vars}{HTML}{FF7F00}        % Orange        - Noms des paramètres et des variables


\fontfamily{Roboto}

\pagecolor{back}
\title{\color{sec1}Fonctions du fichier de librairie \color{path}Checkings.lib}\color{text}
\author{Dimitri OBEID}
\date{2021}
\pagestyle{fancy}

\pdfinfo{
  /Title    (Fonctions du fichier de librairie Checkings.lib)
  /Author   (Dimitri OBEID)
  /Creator  (Dimitri OBEID)
  /Producer (Dimitri OBEID)
  /Subject  (Fonctions du fichier de librairie Checkings.lib)
  /Keywords ()
}

% Mise en page des paragraphes.
\setlength{\parskip}{1em}

\begin{document}
\maketitle
\newpage

\hypertarget{contents}{}
\tableofcontents
\newpage

\color{sec1}
\section{Présentation}\color{text}

\begin{justify}
    Ce fichier de configurations liste chaque variable globale de statut utilisée par la librairie, avec leurs valeurs par défaut, qui sont chargées en mémoire lors de l'inclusion du fichier \textbf{\color{path}Initialization.sh} appartenant au module principal.
\end{justify}

% ----------------------

% -----------------------------------------------

% /////////////////////////////////////////////////////////////////////////////////////////////// %

\color{sec1}\par\noindent\rule{\textwidth}{0.4pt}\color{text}

\color{sec1}
\section{Liste des variables}\color{text}

\begin{justify}
    \textit{Description de la variable :}
\end{justify}

\begin{justify}
     \begin{itemize}
        \item
        {
            \begin{tabular}{l|l|l}
                \textbf{Nom de la variable} & \textbf{Type} & \textbf{Valeurs acceptées}
            \end{tabular}
        }
    \end{itemize}
\end{justify}

% ------------

\par\noindent\rule{\textwidth}{0.4pt}

\begin{justify}
    Cette variable de statut sert à lancer un déboguage de certaines fonctionnalités (dans une section de code écrite de préférence au début du script principal), sans attendre l'atteinte à cette fonctionnalité si le code de cette dernière est écrit trop loin dans un script.
\end{justify}

\begin{justify}
     \begin{itemize}
        \item
        {
            \begin{tabular}{l|l|l}
                \textbf{\color{vars}\$\_\_BU\_MAIN\_STAT\_DEBUG}  & Bool & \textbf{false} || \textbf{true} \\[1\baselineskip]
            \end{tabular}
        }
    \end{itemize}
\end{justify}

% ------------

\par\noindent\rule{\textwidth}{0.4pt}

\begin{justify}
    Cette variable de statut sert à autoriser, empêcher \textbf{OU} restreindre l'exécution de la fonction de décoration de texte avancée \textbf{\color{func}Decho}, en fonction des situations où l'appel de cette fonctions peut provoquer ou non une boucle infinie.
\end{justify}

\begin{justify}
     \begin{itemize}
        \item
        {
            \begin{tabular}{l|l|l}
                \textbf{\color{vars}\$\_\_BU\_MAIN\_STAT\_DECHO}  & String    & \textbf{authorize} || \textbf{forbid} || \textbf{restrict}\\[1\baselineskip]
            \end{tabular}
        }
    \end{itemize}
\end{justify}

% ------------

\par\noindent\rule{\textwidth}{0.4pt}

\begin{justify}
    Cette variable de statut sert à totalement autoriser \textbf{OU} empêcher l'exécution de la fonction \textbf{\color{func}BU::CheckProjectLogStatus}, notamment dans le cas où une fonction d'affichage de texte formaté est appelée à l'intérieur de cette fonction, où via une autre fonction appelée dans le même cas.
\end{justify}

\begin{justify}
     \begin{itemize}
        \item
        {
            \begin{tabular}{l|l|l}
                \textbf{\color{vars}\$\_\_BU\_MAIN\_STAT\_ECHO}   & Bool      & \textbf{false} || \textbf{true}\\[1\baselineskip]
            \end{tabular}
        }
    \end{itemize}
\end{justify}

% ------------

\par\noindent\rule{\textwidth}{0.4pt}

\begin{justify}
    Cette variable de statut sert à déterminer la gravité d'une erreur selon le contexte. Ce choix est à déterminer par l'utilisateur, et la valeur de cette variable est traitée par la fonction \textbf{\color{func}BU::Main::Errors::HandleErrors()} du fichier de librairie \textbf{\color{lime}Checkings.lib}.
\end{justify}

\begin{justify}
     \begin{itemize}
        \item
        {
            \begin{tabular}{l|l|l}
                \textbf{\color{vars}\$\_\_BU\_MAIN\_STAT\_ERROR}  & String    & \textit{aucune valeur} || \textbf{fatal}\\[1\baselineskip]
            \end{tabular}
        }
    \end{itemize}
\end{justify}

% ------------

\par\noindent\rule{\textwidth}{0.4pt}

\begin{justify}
    Cette variable de statut sert à déterminer si la librairie est en phase d'initialisation des modules.
\end{justify}

\begin{justify}
     \begin{itemize}
        \item
        {
            \begin{tabular}{l|l|l}
                \textbf{\color{vars}\$\_\_BU\_MAIN\_STAT\_INITIALIZING}       & Bool  & \textbf{false} || \textbf{true}\\[1\baselineskip]
            \end{tabular}
        }
    \end{itemize}
\end{justify}

% ------------

\par\noindent\rule{\textwidth}{0.4pt}

\begin{justify}
    Cette variable de statut sert à autoriser ou non la création d'un fichier de logs pour le projet en cours.
\end{justify}

\begin{justify}
     \begin{itemize}
        \item
        {
            \begin{tabular}{l|l|l}
                \textbf{\color{vars}\$\_\_BU\_MAIN\_STAT\_LOG}    & Bool      & \textbf{false} || \textbf{true}\\[1\baselineskip]
            \end{tabular}
        }
    \end{itemize}
\end{justify}

% ------------

\par\noindent\rule{\textwidth}{0.4pt}

\begin{justify}
    Cette variable vérifie si un message peut être affiché à l'écran, redirigé vers un fichier de logs, ou les deux à la fois.
\end{justify}

\begin{justify}
     \begin{itemize}
        \item
        {
            \begin{tabular}{l|l|l}
                \textbf{\color{vars}\$\_\_BU\_MAIN\_STAT\_LOG\_REDIRECT}  & String & \textit{aucune valeur} || \textbf{log} || \textbf{tee}\\[1\baselineskip]
            \end{tabular}
        }
    \end{itemize}
\end{justify}

% ------------

\par\noindent\rule{\textwidth}{0.4pt}

\begin{justify}
    Cette variable vérifie si une opération sur un fichier ou un dossier localisé à la racine du système de fichiers peut être effectuée.
\end{justify}

\begin{justify}
     \begin{itemize}
        \item
        {
            \begin{tabular}{l|l|l}
                \textbf{\color{vars}\$\_\_BU\_MAIN\_STAT\_OPERATE\_ROOT}  & String & \textbf{authorized} || \textbf{forbidden} || \textbf{restricted}\\[1\baselineskip]
            \end{tabular}
        }
    \end{itemize}
\end{justify}

% ------------

\par\noindent\rule{\textwidth}{0.4pt}

\begin{justify}
    Cette variable sert à mettre en pause l'exécution du script pendant un temps bref lors de l'affichage d'un header.
\end{justify}

\begin{justify}
     \begin{itemize}
        \item
        {
            \begin{tabular}{l|l}
                \textbf{\color{vars}\$\_\_BU\_MAIN\_STAT\_TIME\_HEADER}   & Float\\[1\baselineskip]
            \end{tabular}
        }
    \end{itemize}
\end{justify}

% ------------

\par\noindent\rule{\textwidth}{0.4pt}

\begin{justify}
    Cette variable sert à mettre en pause l'exécution du script pendant un temps bref lors de l'affichage d'un saut de ligne.
\end{justify}

\begin{justify}
     \begin{itemize}
        \item
        {
            \begin{tabular}{l|l}
                \textbf{\color{vars}\$\_\_BU\_MAIN\_STAT\_TIME\_NEWLINE}  & Float\\[1\baselineskip]
            \end{tabular}
        }
    \end{itemize}
\end{justify}

% ------------

\par\noindent\rule{\textwidth}{0.4pt}

\begin{justify}
    Cette variable sert à mettre en pause l'exécution du script pendant un temps bref lors de l'affichage d'un message à l'écran.
\end{justify}

\begin{justify}
     \begin{itemize}
        \item
        {
            \begin{tabular}{l|l}
                \textbf{\color{vars}\$\_\_BU\_MAIN\_STAT\_TIME\_TXT}  & Float\\[1\baselineskip]
            \end{tabular}
        }
    \end{itemize}
\end{justify}

% ------------

\par\noindent\rule{\textwidth}{0.4pt}

\begin{justify}
    Cette variable de statut vérifie que le script ait la permission de formater le texte (couleurs, gras, italique, soulignage, clignotement, etc...).
\end{justify}

\begin{justify}
     \begin{itemize}
        \item
        {
            \begin{tabular}{l|l|l}
                \textbf{\color{vars}\$\_\_BU\_MAIN\_STAT\_TXT\_FMT}   & Bool & \textbf{false} || \textbf{true}\\[1\baselineskip]
            \end{tabular}
        }
    \end{itemize}
\end{justify}

% ------------

\par\noindent\rule{\textwidth}{0.4pt}

\begin{justify}
\textbf{ATTENTION : Cette variable est dépreciée et sera déplacée dans le module \textbf{\color{lime}os\_specific} dans une prochaine mise à jour}
\end{justify}

\begin{justify}
     \begin{itemize}
        \item
        {
            \begin{tabular}{l|l|l}
                \textbf{\color{vars}\$\_\_BU\_MAIN\_STAT\_USER\_OS}   & &\\[1\baselineskip]
            \end{tabular}
        }
    \end{itemize}
\end{justify}

\end{document}
