\documentclass[a4paper,10pt]{article}

\usepackage[utf8]{inputenc}     % Text encoding (accented characters).
\usepackage[english]{babel}     % Document language.
\usepackage[sfdefault]{roboto}  % Font used in the document.
\usepackage[T1]{fontenc}		% Hyphenation rule for accented characters (for the LaTeX compiler).

\usepackage{fancyhdr}           % Adding headers and footers to each page of the document.
\usepackage{hyperref}           % Creation of clickable links pointing to other parts of the document.
\usepackage{parskip}            % Definable line breaks between each paragraph.

% Don't forget to change the "svgnames" option (colors defined on the RGB model, better for digital display)
% into "dvipsnames" option (based on the CMYK model, better for printing) in the script which converts each
% LaTeX document to printable format.
\usepackage[usenames,svgnames]{xcolor}      % Text colouring.
\usepackage{verbatim}						% Paragraph layout.

\usepackage[document]{ragged2e}								% Text justification.
\usepackage[a4paper,margin=1in,footskip=0.25in]{geometry}	% Document layout.

% --------------------------------------------------------------------------------------------------
% List of defined colors (for the layout and for changing the theme for printing the documentation).

% Color definition                      % Normal | Printable    - Description

\definecolor{back}{HTML}{000000}        % Black | White - Color of the document background.
\definecolor{case}{HTML}{fcff00}        % Yellow        - Color of the "case" conditions.
\definecolor{cmds}{HTML}{909090}        % Gray          - Color of the names of the system commands with their arguments.
\definecolor{cond}{HTML}{be480a}        % Brick         - Color of the "if" conditions.
\definecolor{func}{HTML}{800080}        % Mauve         - Color of the functions defined in each module of the Bash Utils framework.

\definecolor{loop}{HTML}{00ffff}        % Cyan          - Color of the loops.
\definecolor{main}{HTML}{8F00FF}        % Violet        - Color of the functions from the main script.
\definecolor{path}{HTML}{bfff00}        % Lime          - Color of the files and folders paths. and folders paths.
\definecolor{sec1}{HTML}{ff0000}        % Red           - Color of the main and first level titles.
\definecolor{sec2}{HTML}{00ff00}        % Green         - Color of the second level titles.

\definecolor{sec3}{HTML}{0060ff}        % Blue          - Color of the third level titles.
\definecolor{text}{HTML}{ffffff}        % White | Black - Color of the normal text.
\definecolor{vars}{HTML}{FF7F00}        % Orange        - Color of the names of the parameters and the variables.

% ----------------------------------------------------------------
% Definition of the font and the background color of the document.

\fontfamily{Roboto}

\pagecolor{back}

% -----------------------------------------------
% Definition of the informations of the document.

\title{\color{sec1}Color code of the official Bash Utils \\framework documentation}\color{text}
\author{Dimitri OBEID}
\pagestyle{fancy}

\pdfinfo{
  /Title    (Color code of the official Bash Utils framework documentation)
  /Author   (Dimitri OBEID)
  /Creator  (Dimitri OBEID)
  /Producer (Dimitri OBEID)
  /Subject  (Color code of the official Bash Utils framework documentation)
  /Keywords ()
}

% -----------------
% Paragraph layout.
\setlength{\parskip}{1em}

% --------------------------
% Beginning of the document.

\begin{document}
\maketitle
\newpage

\color{sec1}
\section{Colors of the documentation into "normal" format}\color{text}

\begin{justify}
  \textbf{\color{case}WARNING :} It is not recommended to print a document in this format, for the toner of your printer and for a better printing quality with the color model CJMB (CYMK) proposed by the printable format.

  To switch to printable format, please run the \textbf{\color{cmds}latex-convert-to-printable.sh} script. It is located in the folder of executable files, but its code has not been written yet.
\end{justify}

\color{text}\par\noindent\rule{\textwidth}{0.4pt}\color{text}

\begin{justify}
  \textbf{NOTE :} Each time you call the \textbf{\textbackslash{color}} command included in the \textbf{xcolor} package, please call the \textbf{\textbackslash{textbf}} native \LaTeX \ command to make the colored string bold, in order to better highlight it.
\end{justify}

\begin{justify}
  To do so, please follow the following rule :

  \begin{itemize}
    \item \textbackslash{textbf\{\textbackslash{color\{name\_of\_the\_defined\_color\_code}\}}\}
  \end{itemize}
\end{justify}

\color{text}\par\noindent\rule{\textwidth}{0.4pt}\color{text}

\begin{justify}
  The list of color codes for the printable version is located below the following list on the next page.
\end{justify}

\begin{justify}
    \begin{tabular}{lll}
        \textbf{Color code} & \textbf{color}        & \textbf{Description}\\\\

        \color{text}back    & \color{text}Black     & \color{text}Color of the document background\\
        \color{case}case    & \color{case}Yellow    & \color{case}Color of the \textbf{"case"} conditions\\
        \color{cmds}cmds    & \color{cmds}Gray      & \color{cmds}Color of the names of the system commands with their arguments\\\\

        \color{cond}cond    & \color{cond}Brick     & \color{cond}Color of the \textbf{"if"} conditions\\
        \color{func}func    & \color{func}Mauve     & \color{func}Color of the functions defined in each module of the Bash Utils framework\\
        \color{loop}loop    & \color{loop}Cyan      & \color{loop}Color of the loops\\\\

        \color{main}main    & \color{main}Violet    & \color{main}Color of the functions from the main script\\
        \color{path}path    & \color{path}Lime      & \color{path}Color of the files and folders paths\\
        \color{sec1}sec1    & \color{sec1}Red       & \color{sec1}Color of the main and first level titles\\\\

        \color{sec2}sec2    & \color{sec2}Green     & \color{sec2}Color of the second level titles\\
        \color{sec3}sec3    & \color{sec3}Blue      & \color{sec3}Color of the third level titles\\
        \color{text}text    & \color{text}White     & \color{text}Color of the normal text\\\\

        \color{vars}vars    & \color{vars}Orange    & \color{vars}Color of the names of the parameters and the variables\\
    \end{tabular}
\end{justify}

\newpage

% ------------

% ----------------------

% -----------------------------------------------

% /////////////////////////////////////////////////////////////////////////////////////////////// %

\color{sec1}\par\noindent\rule{\textwidth}{0.4pt}\color{text}

\color{red}
\section{Colors of the documentation into the printable format}\color{text}

\begin{justify}
    \begin{tabular}{lll}
        \textbf{Color code} & \textbf{Color}        & \textbf{Description}\\\\

        \color{text}back    & \color{text}White     & \color{text}Color of the document background\\
        \color{case}case    & \color{case}Yellow    & \color{case}Color of the \textbf{"case"} conditions\\
        \color{cmds}cmds    & \color{cmds}Gray      & \color{cmds}Color of the names of the system commands with their arguments\\\\

        \color{cond}cond    & \color{cond}Brick     & \color{cond}Color of the \textbf{"if"} conditions\\
        \color{func}func    & \color{func}Mauve     & \color{func}Color of the functions defined in each module of the Bash Utils framework\\
        \color{loop}loop    & \color{loop}Cyan      & \color{loop}Color of the loops\\\\

        \color{main}main    & \color{main}Violet    & \color{main}Color of the functions from the main script\\
        \color{path}path    & \color{path}Lime      & \color{path}Color of the files and folders paths\\
        \color{sec1}sec1    & \color{sec1}Red       & \color{sec1}Color of the main and first level titles\\\\

        \color{sec2}sec2    & \color{sec2}Green     & \color{sec2}Color of the second level titles\\
        \color{sec3}sec3    & \color{sec3}Blue      & \color{sec3}Color of the third level titles\\
        \color{text}text    & \color{text}Black     & \color{text}Color of the normal text\\\\

        \color{vars}vars    & \color{vars}Orange    & \color{vars}Color of the names of the parameters and the variables\\
    \end{tabular}
\end{justify}


\end{document}
