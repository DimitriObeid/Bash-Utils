\documentclass[a4paper,10pt]{article}

\usepackage[utf8]{inputenc}             % Encodage du texte (caractères accentués).
\usepackage[french]{babel}              % Langue du document.
\usepackage[sfdefault]{roboto}          % Police d'écriture utilisée dans le document.
\usepackage[T1]{fontenc}                % Règle de césure pour les caractères accentués (pour le compilateur LaTeX).

\usepackage{fancyhdr}                   % Ajout d'en-têtes et de pieds de page à chaque page du document.
\usepackage{graphicx}                   % Ajout d'images dans le document.
\usepackage{hyperref}                   % Création de liens cliquables pointant vers d'autres parties du document.
\usepackage{parskip}                    % Sauts de ligne déterminables entre chaque paragraphe.

% Ne pas oublier de modifier l'option "svgnames" (couleurs définies sur le modèle RGB, meilleur pour
% l'affichage numérique) en option "dvipsnames" (basé sur le modèle de couleur CJMB (CMYK), meilleur
% pour l'impression) via le script de conversion en format imprimable.

\usepackage[usenames,dvipsnames]{xcolor}                      % Coloration du texte.
\usepackage{verbatim}                                       % Mise en page des paragraphes.

\usepackage[document]{ragged2e}                             % Justification du texte.
\usepackage[a4paper,margin=1in,footskip=0.25in]{geometry}   % Mise en page du document.


% ----------------------------------------------------------------------------------------------------------------
% Pour plus d'informations et de documentation sur ces paquets LaTeX, veuillez vous référer aux liens ci-dessous :

% inputenc  : https://www.ctan.org/pkg/inputenc
% babel     : https://www.ctan.org/pkg/babel
% roboto    : https://www.ctan.org/pkg/roboto
% fontenc   : https://www.ctan.org/pkg/fontenc

% fancyhdr  : https://www.ctan.org/pkg/fancyhdr
% graphicx  : https://www.ctan.org/pkg/graphicx
% hyperref  : https://www.ctan.org/pkg/hyperref
% parskip   : https://www.ctan.org/pkg/parskip

% xcolor    : https://www.ctan.org/pkg/xcolor
% verbatim  : https://www.ctan.org/pkg/verbatim

% ragged2e  : https://www.ctan.org/pkg/ragged2e
% geometry  : https://www.ctan.org/pkg/geometry


% ------------------------------------------------------------------------------------------------------------------------
% Liste des couleurs définies (pour la mise en page et pour le changement de thème pour l'impression de la documentation).

% Définition de la couleur       % Normal   | Imprimable    - Description

\definecolor{back}{HTML}{000000} % Noir     | Blanc         - Couleur de fond du document.
\definecolor{case}{HTML}{fcff00} % Jaune                    - Couleur des conditions "case".
\definecolor{cmds}{HTML}{909090} % Gris                     - Couleur des noms de commandes du système et de leurs arguments.
\definecolor{cond}{HTML}{be480a} % Brique                   - Couleur des conditions "if".
\definecolor{func}{HTML}{800080} % Mauve                    - Couleur des fonctions définies dans chaque module du framework Bash Utils.

\definecolor{loop}{HTML}{00ffff} % Cyan                     - Couleur des boucles.
\definecolor{main}{HTML}{8F00FF} % Violet                   - Couleur des fonctions du script principal.
\definecolor{path}{HTML}{bfff00} % Citron vert              - Couleur des chemins de dossiers et de fichiers.
\definecolor{sec1}{HTML}{ff0000} % Rouge                    - Couleur des titres principaux et de premier niveau.
\definecolor{sec2}{HTML}{00ff00} % Vert                     - Couleur des titres de deuxième niveau.

\definecolor{sec3}{HTML}{0060ff} % Bleu                     - Couleur des titres de troisième niveau.
\definecolor{text}{HTML}{ffffff} % Blanc    | Noir          - Couleur du texte normal.
\definecolor{vars}{HTML}{FF7F00} % Orange                   - Couleur des noms des paramètres et des variables.


% ------------------------------------------------------------------------
% Définition de la police d'écriture et de la couleur de fond du document.

\fontfamily{Roboto}

\pagecolor{back}


% ----------------------------------------
% Définition des informations du document.

\title{\color{sec1}Fonctions du fichier de librairie \color{path}Errors.lib}\color{text}
\author{Dimitri OBEID}
\date{2021}
\pagestyle{fancy}

\pdfinfo{
  /Title    (Fonctions du fichier de librairie Errors.lib)
  /Author   (Dimitri OBEID)
  /Creator  (Dimitri OBEID)
  /Producer (Dimitri OBEID)
  /Subject  (Fonctions du fichier de librairie Errors.lib)
  /Keywords ()
}


% ------------------------------------------------------------
% Mise en page des paragraphes et des en-têtes de chaque page.

\setlength{\parskip}{1em}

\setlength{\headheight}{13pt}


% ------------------
% Début du document.

\begin{document}
    \maketitle
    \tableofcontents
    \newpage

    \color{sec1}
    \section{Présentation}\color{text}

    \color{sec2}
    \subsection{Description}\color{text}

    \begin{justify}
        Ce fichier source inclut des fonctions servant à la gestion d'erreurs.
    \end{justify}

    % ------------

    % ----------------------

    % -----------------------------------------------

    \color{sec2}\par\noindent\rule{\textwidth}{0.4pt}\color{text}

    \color{sec2}
    \subsection{Définitions des éléments mentionnés}\color{text}

    \color{sec3}
    \subsubsection{Chemins des fichiers}\color{text}

    \textbf{Nom du fichier : \color{path}}\\[1\baselineskip]

    \textbf{Dossier parent : \color{path}}\\[1\baselineskip]

    \begin{justify}
        \textbf{*} : Voir la variable \textbf{\color{vars}\$\_\_BU\_MAIN\_ROOT\_DIR\_PATH\color{path}/lib/functions/main} mentionnée dans la sous-sous-catégorie \textbf{\color{sec3}Variables globales externes et / ou variables d'environnement appelées}.
    \end{justify}

    % ------------

    % ----------------------

    \color{path}\par\noindent\rule{\textwidth}{0.4pt}\color{text}\\[1\baselineskip]


    \color{sec3}
    \subsubsection{Fonctions}\color{text}

    \textbf{Fonction : \color{func}}\\[1\baselineskip]

    \textbf{Fichier de définition :} \textbf{\color{path}}\\[1\baselineskip]

    % ------------

    % ----------------------

    \color{sec3}\par\noindent\rule{\textwidth}{0.4pt}\color{text}

    \color{sec3}
    \subsubsection{Variables globales externes et / ou variables d'environnement appelées}\color{text}

    \textbf{Variable : \color{vars}}\\[1\baselineskip]

    \textbf{Description :} .\\[1\baselineskip]

    \textbf{Fichier de définition : \color{path}}\\[1\baselineskip]


    % ------------

    \par\noindent\rule{\textwidth}{0.4pt}\\[1\baselineskip]

    \textbf{Variable : \color{vars}}\\[1\baselineskip]

    \textbf{Description :} .\\[1\baselineskip]

    \textbf{Fichier de définition : \color{path}}\\[1\baselineskip]

    % ------------

    % ----------------------

    % -----------------------------------------------

    % /////////////////////////////////////////////////////////////////////////////////////////////// %

    \color{sec1}\par\noindent\rule{\textwidth}{0.4pt}\color{text}

    \color{sec1}
    \section{Fonctions - Section « ERRORS HANDLING FUNCTIONS »}\color{text}

    \color{sec2}
    \subsection{Sous-section « CLASSIC ERRORS HANDLING »}\color{text}

    \color{sec3}
    \subsubsection{BU.Main.Errors.HandleSmallErrors}\color{text}

    \begin{justify}
        \textbf{Description :}\\
        Cette fonction affiche un simple message d'avertissement ou d'erreur, selon le contexte choisi, contrairement à la fonction \textbf{\color{func}BU.Main.Errors.HandleErrors}, qui fait un rapport complet de l'erreur, et qui est à priviliégier pour de plus grosses erreurs.
    \end{justify}

    \begin{justify}
        \textbf{Paramètres :}

        \begin{itemize}
            \item \textbf{\color{vars}p\_type} Ce paramètre attend soit le caractère \textbf{W} (warning), soit le paramètre \textbf{E} (error)\\\mbox{}

            \item \textbf{\color{vars}p\_string} Ce paramètre attend la chaîne de caractère à afficher\\\mbox{}

            \item \textbf{\color{vars}p\_return} Ce paramètre attend les mêmes caractères que le paramètre \textbf{\color{vars}p\_type}\\\mbox{}

            \item \textbf{\color{vars}p\_cpls} Ce paramètre attend la chaîne de caractère \textbf{CPLS}, pour changer la valeur de la variable globale de statut \textbf{\color{vars}\$\_\_BU\_MAIN\_STAT\_ECHO}.\\\mbox{}

            \item \textbf{Note : CPLS} sont les initiales de la fonction \textbf{\color{func}BU.Main.Status.CheckProjectLogStatus}, qui vérifie si et comment un message peut être redirigé vers un fichier de logs, et où un rappel d'une fonction \textbf{\color{func}BU.Echo<...>} avec la valeur de la variable globale de statut mentionnée ci-dessus à \textbf{false} provoquerait une boucle infinie.
        \end{itemize}
    \end{justify}

    \begin{justify}
        \textbf{Fonctionnement :}\\
        Cette fonction vérifie en premier lieu si la valeur \textbf{CPLS} est passée en quatrième argument, pour sauvegarder la valeur actuelle de la variable globale de statut \textbf{\color{vars}\$\_\_BU\_MAIN\_STAT\_ECHO}, et la mettre à \textbf{true}.
    \end{justify}

    \begin{justify}
        Passer cette valeur à ce paramètre est absolument nécessaire dans les situations où l'appel des fonctions \textbf{\color{func}BU.Echo<...>()} peut provoquer une boucle infinie.
    \end{justify}

    \begin{justify}
        Ensuite, elle vérifie quelle valeur a été passée en premier argument, puis affiche la chaîne de caractères passée en deuxième argument selon la valeur du premier (W = warning, donc appel de la fonction \textbf{\color{func}BU.Main.Echo.Warning}, et E = error, donc appel de la fonction \textbf{\color{func}BU.Main.Echo.Error}).
    \end{justify}

    \begin{justify}
        Si la valeur du premier argument n'est ni renseignée, ni attendue, alors une de ces fonctions est aléatoirement appelée pour afficher cette chaîne de caractères.
    \end{justify}

    \begin{justify}
        Une fois la chaîne de caractère affichée, la fonction vérifie quelle valeur a été passée en troisième argument (\textbf{\color{vars}p\_return}), pour simplement continuer l'exécution de votre script ou pour directement arrêter son exécution.
    \end{justify}

    \begin{justify}
        En cas d'absence de valeur ou de présence d'une valeur inattendue, un message d'avertissement s'affiche sur le terminal, avant que l'exécution du script ne soit définitivement interrompue.
    \end{justify}

    \begin{justify}
        Enfin, si la valeur \textbf{CPLS} a été passée en quatrième argument, l'ancienne valeur est automatiquement réassignée à la variable globale de statut \textbf{\color{vars}\$\_\_BU\_MAIN\_STAT\_ECHO}.
    \end{justify}

    \begin{justify}
        \textbf{Utilisation :}
    \end{justify}
\end{document}
